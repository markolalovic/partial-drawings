\documentclass{beamer}

\mode<presentation> {
\usetheme{Madrid}
}
\usepackage[utf8x]{inputenc}
\usepackage[english]{babel}
\usepackage[labelformat=empty]{caption}

\usepackage{graphicx}
\usepackage{booktabs}

% for a tick
\usepackage{bbding}
\usepackage{pifont}
\usepackage{wasysym}
\usepackage{amssymb}

\usepackage{tikz}
\usetikzlibrary{calc}
\usetikzlibrary{positioning}
\usetikzlibrary{arrows}

% subfigure
\usepackage{caption}
\usepackage{subcaption}
\usetikzlibrary{shapes}
\usetikzlibrary{decorations.markings}
\usepackage{accents}

% angles
\usepackage{tkz-euclide}

\definecolor{blue}{RGB}{51,105,232}
\definecolor{yellow}{RGB}{238,118,0}
\definecolor{orange}{RGB}{213,15,37}
\definecolor{red}{RGB}{139,37,0}
\definecolor{grey}{RGB}{211,211,211} 
\definecolor{gray}{RGB}{211,211,211} 
\definecolor{green}{RGB}{0,153,37}
\definecolor{color-trapezoid}{RGB}{211,211,211} 
\definecolor{color-inside}{RGB}{211,211,211} 
\definecolor{color-parallelogram}{RGB}{211,211,211} 
\definecolor{color-CL}{RGB}{211,211,211}

% trapezoid tpz
\newcommand\tpz{%
  \ensuremath{
    \begin{tikzpicture}[line width=0.13ex] %0.15ex
      \pgfsetbaseline{-0.575ex}
      \useasboundingbox (-1ex, -1ex) rectangle (1ex, 1ex);
      \draw (-0.7ex, -0.5ex ) -- (0.7ex, -0.5 ex) -- (0.4 ex, 0.5ex) -- (-0.4 ex, 0.5ex) -- cycle;
    \end{tikzpicture}%
  }
}

\newcommand{\abovesym}[1]{\tilde{#1}}
\newcommand{\undersym}[1]{\underaccent{\tilde}{#1}}
\newcommand{\rectansym}{\Box}
\newcommand{\quadsym}{\Diamond}
\newcommand{\trisym}{\triangle}

\newcommand{\norm}[1]{\left\lVert #1 \right\rVert}
\newcommand{\sangle}{\sphericalangle}

\newcommand{\threepartdef}[6]
{
	\left\{
		\begin{array}{lll}
			#1 & \mbox{if } #2 \\
			#3 & \mbox{if } #4 \\
			#5 & \mbox{else } #6
		\end{array}
	\right.
}
\newcommand{\twopartdef}[4]
{
	\left\{
		\begin{array}{ll}
			#1 & \mbox{if } #2 \\
			#3 & \mbox{if } #4
		\end{array}
	\right.
}
\newcommand{\vertwopartdef}[4]
{
	\left\{
		\begin{array}{ll}
			#1 & \mbox{if } #2 \\[22pt]
			#3 & \mbox{if } #4
		\end{array}
	\right.
}

% floor
\usepackage{mathtools}
\DeclarePairedDelimiter\ceil{\lceil}{\rceil}
\DeclarePairedDelimiter\floor{\lfloor}{\rfloor}

% proj
\DeclareMathOperator{\proj}{proj}

%----------------------------------------------------------------------------------------
%	TITLE PAGE
%----------------------------------------------------------------------------------------
\begin{document}

\title[Partial Drawings of Complete Graphs]{Partial Drawings of Complete Graphs}

\author[Marko Lalović]{
Marko Lalović \\
Mentor: prof.\ dr.\ Gašper Fijavž}

\institute[UL]{
University of Ljubljana \\
Faculty of Computer and Information Science \\
Faculty of Mathematics and Physics}
\date{{Nov 2014}}

\begin{frame}
\titlepage
\end{frame}

\begin{frame}
\frametitle{Outline}
\tableofcontents 
\end{frame}

%----------------------------------------------------------------------------------------
%	PRESENTATION SLIDES
%----------------------------------------------------------------------------------------

%------------------------------------------------
\section{Idea of Partial Drawings}

%------------------------------------------------
\begin{frame}
\frametitle{Only Two Obstacles to Planarity}
\begin{figure}[H]
\centering
\resizebox{8cm}{!}{%
\begin{tikzpicture}[node distance=0.1cm,>=latex,scale=1.6, dot/.style={circle,inner sep=1pt,fill,label={#1}, name=#1},
  extended line/.style={shorten >=-#1,shorten <=-#1},
 extended line/.default=1cm]  

\node [dot=] at (4.01,4.78) {};
\node [dot=] at (4,2.76){};
\node [dot=] at (5.67,5.82){};
\node [dot=] at (6.34,2.6){};
\node [dot=] at (7.44,4.87){};
\node [dot=] at (6.41,4){};
\node [dot=] at (7.65,4) {};
\node [dot=] at (9,4) {};
\node [dot=] at (7.68,2.6) {};
\node [dot=] at (9.03,2.6) {};
\node [dot=] at (8.55,5.37) {};
\node [dot=] at (5.25,2.05) {};
\node [dot=] at (7,2) {};

\draw [thick] (4.01,4.78)-- (5.67,5.82);
\draw [thick] (4.01,4.78)-- (4,2.76);
\draw [thick] (4,2.76)-- (6.41,4);
\draw [thick] (4,2.76)-- (7.44,4.87);
\draw [thick] (4,2.76)-- (5.67,5.82);
\draw [thick] (4,2.76)-- (6.34,2.6);
\draw [thick] (5.25,2.05)-- (4.01,4.78);
\draw [thick] (5.25,2.05)-- (6.41,4);
\draw [thick] (4.01,4.78)-- (6.34,2.6);
\draw [thick] (6.34,2.6)-- (7,2);
\draw [thick] (4.01,4.78)-- (6.41,4);
\draw [thick] (4.01,4.78)-- (7.44,4.87);
\draw [thick] (5.67,5.82)-- (6.34,2.6);
\draw [thick] (6.34,2.6)-- (6.41,4);
\draw [thick] (6.41,4)-- (5.67,5.82);
\draw [thick] (5.67,5.82)-- (7.44,4.87);
\draw [thick] (7.44,4.87)-- (6.41,4);
\draw [thick] (6.41,4)-- (7.68,2.6);
\draw [thick] (7.68,2.6)-- (7.65,4);
\draw [thick] (7.65,4)-- (7,2);
\draw [thick] (7,2)-- (9.03,2.6);
\draw [thick] (9.03,2.6)-- (6.41,4);
\draw [thick] (6.41,4)-- (6.34,2.6);
\draw [thick] (6.34,2.6)-- (7.44,4.87);
\draw [thick] (7.44,4.87)-- (8.55,5.37);
\draw [thick] (8.55,5.37)-- (5.67,5.82);
\draw [thick] (8.55,5.37)-- (9.03,2.6);
\draw [thick] (8.55,5.37)-- (7.65,4);
\draw [thick] (7.65,4)-- (6.34,2.6);
\draw [thick] (6.34,2.6)-- (9,4);
\draw [thick] (9,4)-- (9.03,2.6);
\draw [thick] (9.03,2.6)-- (7.65,4);
\draw [thick] (7.68,2.6)-- (9,4);

\end{tikzpicture}
}
\caption{Drawing of a graph that contains subgraphs $K_{5}$ and $K_{3,3}$.}
\end{figure}
\footnotesize{Kuratowski’s theorem.}
\end{frame}
%------------------------------------------------

%------------------------------------------------
\begin{frame}
\frametitle{Idea of Partial Drawings}
\begin{figure}[H]
\centering
\resizebox{8cm}{!}{%
\begin{tikzpicture}[node distance=0.1cm,>=latex,scale=1.6, dot/.style={circle,inner sep=1pt,fill,label={#1}, name=#1},
  extended line/.style={shorten >=-#1,shorten <=-#1},
 extended line/.default=1cm]  
 
\node [dot=] at (4.01,4.78) {};
\node [dot=] at (4,2.76){};
\node [dot=] at (5.67,5.82){};
\node [dot=] at (6.34,2.6){};
\node [dot=] at (7.44,4.87){};
\node [dot=] at (6.41,4){};
\node [dot=] at (7.65,4) {};
\node [dot=] at (9,4) {};
\node [dot=] at (7.68,2.6) {};
\node [dot=] at (9.03,2.6) {};
\node [dot=] at (8.55,5.37) {};
\node [dot=] at (5.25,2.05) {};
\node [dot=] at (7,2) {};

\draw [thick] (4.01,4.78)-- (4.01,4.28);
\draw [thick] (4.01,3.26)-- (4,2.76);
\draw [thick] (5.67,5.82)-- (5.83,5.02);
\draw [thick] (6.17,3.41)-- (6.34,2.6);
\draw [thick] (7.44,4.87)-- (7.18,4.65);
\draw [thick] (6.67,4.22)-- (6.41,4);
\draw [thick] (4.01,4.78)-- (4.61,4.59);
\draw [thick] (5.81,4.19)-- (6.41,4);
\draw [thick] (4,2.76)-- (4.86,3.28);
\draw [thick] (6.58,4.34)-- (7.44,4.87);
\draw [thick] (4,2.76)-- (4.59,2.72);
\draw [thick] (5.75,2.64)-- (6.34,2.6);
\draw [thick] (6.34,2.6)-- (6.36,2.95);
\draw [thick] (6.39,3.65)-- (6.41,4);
\draw [thick] (4.01,4.78)-- (4.43,5.04);
\draw [thick] (5.25,5.56)-- (5.67,5.82);
\draw [thick] (5.67,5.82)-- (6.11,5.59);
\draw [thick] (7,5.11)-- (7.44,4.87);
\draw [thick] (4.01,4.78)-- (4.87,4.81);
\draw [thick] (6.58,4.85)-- (7.44,4.87);
\draw [thick] (4.01,4.78)-- (4.59,4.24);
\draw [thick] (5.76,3.15)-- (6.34,2.6);
\draw [thick] (4.42,3.52)-- (4,2.76);
\draw [thick] (5.25,5.06)-- (5.67,5.82);
\draw [thick] (4,2.76)-- (4.6,3.07);
\draw [thick] (5.81,3.69)-- (6.41,4);
\draw [thick] (6.61,3.17)-- (6.34,2.6);
\draw [thick] (7.17,4.3)-- (7.44,4.87);
\draw [thick] (5.67,5.82)-- (5.85,5.37);
\draw [thick] (6.22,4.45)-- (6.41,4);
\draw [thick] (6.41,4)-- (6.73,3.65);
\draw [thick] (7.36,2.95)-- (7.68,2.6);
\draw [thick] (6.34,2.6)-- (6.67,2.95);
\draw [thick] (7.65,4)-- (7.32,3.65);
\draw [thick] (7.65,4)-- (8,3.65);
\draw [thick] (8.69,2.95)-- (9.03,2.6);
\draw [thick] (7.68,2.6)-- (7.67,2.95);
\draw [thick] (7.65,4)-- (7.66,3.65);
\draw [thick] (7.68,2.6)-- (8.01,2.95);
\draw [thick] (8.67,3.65)-- (9,4);
\draw [thick] (9,4)-- (9.01,3.65);
\draw [thick] (9.03,2.95)-- (9.03,2.6);
\draw [thick] (6.41,4)-- (7.07,3.65);
\draw [thick] (8.38,2.95)-- (9.03,2.6);
\draw [thick] (6.34,2.6)-- (7,2.95);
\draw [thick] (8.33,3.65)-- (9,4);
\draw [thick] (5.67,5.82)-- (6.39,5.71);
\draw [thick] (7.83,5.48)-- (8.55,5.37);
\draw [thick] (8.55,5.37)-- (8.67,4.68);
\draw [thick] (8.91,3.29)-- (9.03,2.6);
\draw [thick] (7.65,4)-- (7.88,4.34);
\draw [thick] (8.33,5.03)-- (8.55,5.37);
\draw [thick] (7.44,4.87)-- (7.72,4.99);
\draw [thick] (8.27,5.24)-- (8.55,5.37);
\draw [thick] (4.01,4.78)-- (4.32,4.1);
\draw [thick] (4.94,2.73)-- (5.25,2.05);
\draw [thick] (5.25,2.05)-- (5.54,2.53);
\draw [thick] (6.12,3.51)-- (6.41,4);
\draw [thick] (7,2)-- (7.51,2.15);
\draw [thick] (8.53,2.45)-- (9.03,2.6);
\draw [thick] (7.65,4)-- (7.49,3.5);
\draw [thick] (7.16,2.5)-- (7,2);
\draw [thick] (6.34,2.6)-- (6.5,2.45);
\draw [thick] (6.83,2.15)-- (7,2);

\end{tikzpicture}
}
\caption{Partial drawing of a graph that contains subgraphs $K_{5}$ and $K_{3,3}$.}
\end{figure}
\footnotesize{User study [M.\ Burch et.\ al., 2012].}
\end{frame}
%------------------------------------------------

%------------------------------------------------
\begin{frame}
\frametitle{Idea of Partial Drawings}
\begin{figure}[H]
\centering
%\resizebox{8cm}{!}
\caption{Calls between locations after the earthquake on 17. October 1989.}
\end{figure}
\footnotesize{[R.\ A.\ Becker et.\ al., 1995].}
\end{frame}
%------------------------------------------------

%------------------------------------------------
\begin{frame}
\frametitle{Idea of Partial Drawings}
\begin{figure}[H]
\centering
%\resizebox{8cm}{!}
\caption{Calls between locations presented with partial edges.}
\end{figure}
\footnotesize{[R.\ A.\ Becker et.\ al., 1995].}
\end{frame}
%------------------------------------------------

%------------------------------------------------
\begin{frame}
\frametitle{What is a Partial Drawing}
Partial edge is a pair of quarter-lines called stubs and we treat them as
closed sets.
\begin{figure}[H]
\centering
\begin{tikzpicture}[node distance=0.1cm,>=latex,scale=0.4, dot/.style={circle,inner sep=1pt,fill,label={#1}, name=#1},
  extended line/.style={shorten >=-#1,shorten <=-#1},
 extended line/.default=1cm]  

\node [dot=] at  (1.01,0.13) {};
\node [dot=] at (8.06,2.44) {};

\draw [thick] (1.01,0.13)-- (2.77,0.7);
\draw [thick] (6.3,1.86)-- (8.06,2.44);

\end{tikzpicture}
\caption{Stubs.}
\end{figure}

In addition, we require that the drawing is without crossings of partial
edges or stubs.
\begin{figure}[H]
\centering
\input{./tikz/no-crossing-requirement.subfig}
\end{figure}
\end{frame}
%------------------------------------------------

\setcounter{subfigure}{0} % reset subfigure counter

%------------------------------------------------
\begin{frame}
\frametitle{What is a Partial Drawing}
Partial drawing depends only on the relative positions of points.
\begin{figure}[H]
\centering
\input{./tikz/various-drawings-k6.subfig}
\caption{Various drawings of $K_{6}$.}
\end{figure}
\end{frame}

%------------------------------------------------
\begin{frame}
\frametitle{What is the Problem We Were Trying to Solve}

\begin{block}{Problem}
For how big complete graph the partial drawing exists?
\end{block}

\begin{block}{In other words}
Let $M$ denote the maximum number of points in a complete graph that we
can draw as partial drawing. We want to estimate the upper bound
of $M$:
$$M < \text{ ?}$$
\end{block}
\begin{figure}[H]
\centering
\input{./tikz/number-line-1.tikz}
\end{figure}
\end{frame}
%------------------------------------------------

\section{Known Work}
%------------------------------------------------
\begin{frame}
\frametitle{Known Work}
The formalization of the problem and estimate of the ``lower bound'' $M \geq 16$ [T.\ Bruckdorfer, M.\ Kaufmann, 2012].

\begin{figure}[H]
\centering
\includegraphics[width=.3\textwidth]{./figures/partial-drawing-of-k_16.pdf}
\caption{Partial drawing of $K_{16}$}
\end{figure}
\begin{figure}[H]
\centering
\begin{tikzpicture}[node distance=0.1cm,>=latex,scale=0.4, dot/.style={circle,inner sep=1pt,fill,label={#1}, name=#1}]

\draw [xshift=1cm] node[circle,fill,inner sep=8pt, color = white, label=below:$...$] {};
\draw [xshift=5cm] node[circle,fill,inner sep=8pt, color = white, label=below:$...$] {};
\draw [xshift=13cm] node[circle,fill,inner sep=5pt, color = white, label=below:$?$] {};

\draw [xshift=15cm] node[circle,fill,inner sep=8pt, color = white, label=below:$...$] {};

\draw[gray,thick,->] ({-1}, 0) -- (19, 0) {};

\tikzstyle{every node}=[draw,shape=circle]

\draw [xshift=0cm] node[circle,fill,color=green,inner sep=2pt,label=below:$1$]{};
\draw [xshift=2cm] node[circle,fill,color=green,inner sep=2pt]{};
\draw [xshift=3cm] node[circle,fill,color=green,inner sep=2pt,label=below:$16$]{};

\draw [xshift=12cm] node[circle,fill,color=green,inner sep=2pt,label=below:$M$]{};
\draw [xshift=13cm] node[circle,fill,color=red,inner sep=2pt]{};
\draw [xshift=14cm] node[circle,fill,color=red,inner sep=2pt]{};
\end{tikzpicture}
\end{figure}
\end{frame}
%------------------------------------------------

%------------------------------------------------
\begin{frame}
\frametitle{Known Work}
It is not possible to draw a partial drawing of the complete graph on
seventeen points which lie in \textcolor{blue}{one-sided convex position} [T.\ Bruckdorfer et.\ al., 2013]. According to the result of Erd\H{o}s and Szekeres we obtain $M < \binom{30}{15} = 155\text{ }117\text{ }520$.

\begin{figure}[H]
\centering
\resizebox{6cm}{!}{%
\begin{tikzpicture}[node distance=0.1cm,>=latex,scale=0.9, dot/.style={circle,inner sep=1pt,fill,label={#1}, name=#1},
extended line/.style={shorten >=-#1,shorten <=-#1},
extended line/.default=1cm]  

\draw [color=blue] (2,2)-- (7,3);
\draw [shift={(4.5,2.5)},color=blue]  plot[domain=0.2:3.34,variable=\t]({1*2.55*cos(\t r)+0*2.55*sin(\t r)},{0*2.55*cos(\t r)+1*2.55*sin(\t r)});
\draw [color=blue] (2.64,4.25)-- (2.48,3.68);
\draw [color=blue] (2.16,2.56)-- (2,2);
\draw [color=blue] (2.24,2.64)-- (2,2);
\draw [color=blue] (2.73,3.91)-- (2.97,4.54);
\draw [color=blue] (2.64,4.25)-- (3.73,3.93);
\draw [color=blue] (5.91,3.31)-- (7,3);
\draw [color=blue] (2.97,4.54)-- (3.98,4.16);
\draw [color=blue] (5.99,3.39)-- (7,3);

\fill [color=blue] (2,2) circle (1pt);
\fill [color=blue] (7,3) circle (1pt);
\fill [color=blue] (2.97,4.54) circle (1pt);
\fill [color=blue] (4,5) circle (1pt);
\fill [color=blue] (5.17,4.96) circle (1pt);
\fill [color=blue] (3.48,4.84) circle (1pt);
\fill [color=blue] (2.64,4.25) circle (1pt);
\fill [color=blue] (4.55,5.05) circle (1pt);
\fill [color=blue] (2.36,3.89) circle (1pt);
\fill [color=blue] (2.18,3.55) circle (1pt);
\fill [color=blue] (2.04,3.16) circle (1pt);
\fill [color=blue] (1.97,2.8) circle (1pt);
\fill [color=blue] (5.61,4.79) circle (1pt);
\fill [color=blue] (6.05,4.52) circle (1pt);
\fill [color=blue] (6.45,4.14) circle (1pt);
\fill [color=blue] (6.81,3.58) circle (1pt);
\fill [color=blue] (1.95,2.41) circle (1pt);
\fill [color=gray] (5.11,1.79) circle (1pt);
\fill [] (3.98,1.46) circle (1pt);
\fill [] (2.72,1.51) circle (1pt);
\fill [] (1.52,1.23) circle (1pt);
\fill [] (3.8,0.81) circle (1pt);
\fill [] (5.28,1.08) circle (1pt);
\fill [] (6.31,1.45) circle (1pt);
\fill [] (5.77,1.75) circle (1pt);
\fill [] (6.09,1.33) circle (1pt);
\fill [] (6.05,1.62) circle (1pt);
\fill [] (6.66,1.84) circle (1pt);
\fill [] (7.01,1.57) circle (1pt);
\fill [] (7.15,0.85) circle (1pt);
\fill [] (4.4,0.63) circle (1pt);
\fill [] (4.43,1) circle (1pt);
\fill [] (4.28,0.76) circle (1pt);
\fill [] (4.06,3.21) circle (1pt);
\fill [] (1,4) circle (1pt);
\fill [] (1.63,4.22) circle (1pt);
\fill [] (1.63,3.76) circle (1pt);
\fill [] (1.54,3.23) circle (1pt);
\fill [] (2,4) circle (1pt);
\fill [] (1.98,4.19) circle (1pt);
\fill [] (2.38,4.67) circle (1pt);
\fill [] (4.63,4.38) circle (1pt);
\fill [] (4.86,3.97) circle (1pt);
\fill [] (5.38,4.2) circle (1pt);
\fill [] (5.39,4.44) circle (1pt);
\fill [] (4.81,4.31) circle (1pt);
\fill [] (4.27,3.93) circle (1pt);
\fill [] (3.47,3.62) circle (1pt);
\fill [] (3.14,3.61) circle (1pt);
\fill [] (3.34,3.45) circle (1pt);
\fill [] (3.07,2.84) circle (1pt);
\fill [] (3.87,2.86) circle (1pt);
\fill [] (5,3) circle (1pt);
\fill [] (5.68,3.28) circle (1pt);
\fill [] (7.41,3.98) circle (1pt);
\fill [] (7.25,4.38) circle (1pt);
\fill [] (7.04,4.73) circle (1pt);
\fill [] (6.38,5.2) circle (1pt);
\fill [] (5.04,5.37) circle (1pt);
\fill [] (3.58,5.4) circle (1pt);
\fill [] (2.31,5.19) circle (1pt);
\fill [] (7.24,2.13) circle (1pt);
\fill [] (7.5,2.64) circle (1pt);
\fill [] (7.55,3.06) circle (1pt);
\fill [] (7.31,3.49) circle (1pt);
\fill [] (5,2) circle (1pt);
\fill [] (4.14,2.05) circle (1pt);
\fill [] (4,2) circle (1pt);
\fill [] (2.82,1.96) circle (1pt);
\fill [] (2.79,0.91) circle (1pt);
\fill [] (2.48,1.53) circle (1pt);
\fill [] (6.28,2.47) circle (1pt);
\fill [] (6.69,2.84) circle (1pt);
\fill [] (6.46,2.66) circle (1pt);
\fill [] (6.59,1.69) circle (1pt);
\fill [] (6.97,1.15) circle (1pt);
\fill [] (7.55,1.72) circle (1pt);
\fill [] (7.73,1.96) circle (1pt);
\fill [] (8.11,2.2) circle (1pt);
\fill [] (7.85,3.29) circle (1pt);
\fill [] (6.67,4.94) circle (1pt);
\fill [] (6.78,4.58) circle (1pt);
\fill [] (5.83,5.35) circle (1pt);
\fill [] (4.59,5.4) circle (1pt);
\fill [] (3.78,5.36) circle (1pt);
\fill [] (3.34,5.14) circle (1pt);
\fill [] (1.43,2.48) circle (1pt);
\fill [] (1.09,3.56) circle (1pt);
\fill [] (1.09,1.83) circle (1pt);
\fill [] (6.14,0.72) circle (1pt);
\fill [] (7.45,0.7) circle (1pt);
\fill [] (5.49,2.41) circle (1pt);
\fill [] (5.12,2.53) circle (1pt);
\fill [] (4.85,2.47) circle (1pt);
\fill [] (4.6,2.72) circle (1pt);
\fill [] (4.11,2.65) circle (1pt);
\fill [] (3.91,2.63) circle (1pt);
\fill [] (3.66,2.59) circle (1pt);
\fill [] (3.4,2.56) circle (1pt);
\fill [] (4.3,3.29) circle (1pt);
\fill [] (4.58,3.43) circle (1pt);
\fill [] (5.12,3.56) circle (1pt);
\fill [] (5.94,3.79) circle (1pt);
\fill [] (1.76,1.04) circle (1pt);
\fill [] (1.89,1.41) circle (1pt);
\fill [] (2.51,3.08) circle (1pt);
\fill [] (2.34,2.97) circle (1pt);
\fill [] (3.61,4.51) circle (1pt);
\fill [] (3,5) circle (1pt);
\fill [] (4.78,5.71) circle (1pt);
\fill [] (1.62,5.33) circle (1pt);
\fill [] (1.55,4.56) circle (1pt);
\fill [] (3.18,5.33) circle (1pt);
\end{tikzpicture}
}
\end{figure}

\begin{figure}[H]
\centering
\input{./tikz/number-line-3.tikz}
\end{figure}
\end{frame}
%------------------------------------------------

%------------------------------------------------
\begin{frame}
\frametitle{Known Work}
[Bruckdorfer et.\ al., 2013]: $M < 241$.
\begin{figure}[H]
\centering
\begin{tikzpicture}[node distance=0.1cm,>=latex,scale=0.4, dot/.style={circle,inner sep=1pt,fill,label={#1}, name=#1}]

\draw [xshift=1cm] node[circle,fill,inner sep=8pt, color = white, label=below:$...$] {};
\draw [xshift=5cm] node[circle,fill,inner sep=8pt, color = white, label=below:$...$] {};
\draw [xshift=16cm] node[circle,fill,inner sep=3pt, color = white, label=below:\footnotesize{$241$}] {};

\draw[gray,thick,->] ({-1}, 0) -- (19, 0) {};

\tikzstyle{every node}=[draw,shape=circle]

\draw [xshift=0cm] node[circle,fill,color=green,inner sep=2pt,label=below:$1$]{};
\draw [xshift=2cm] node[circle,fill,color=green,inner sep=2pt]{};
\draw [xshift=3cm] node[circle,fill,color=green,inner sep=2pt,label=below:$16$]{};

\draw [xshift=16cm] node[circle,fill,color=red,inner sep=2pt]{};
\draw [xshift=17cm] node[circle,fill,color=red,inner sep=2pt]{};
\draw [xshift=18cm] node[circle,fill,color=red,inner sep=2pt]{};
%,label=below:$155 117 520$
\end{tikzpicture}
\end{figure}
\vspace{2.7cm}
\end{frame}
%------------------------------------------------

\section{Our Contribution}
%------------------------------------------------
\begin{frame}
\frametitle{Our Contribution}
[Bruckdorfer et.\ al., 2013]: $M < 241$.
\begin{figure}[H]
\centering
\begin{tikzpicture}[node distance=0.1cm,>=latex,scale=0.4, dot/.style={circle,inner sep=1pt,fill,label={#1}, name=#1}]

\draw [xshift=1cm] node[circle,fill,inner sep=8pt, color = white, label=below:$...$] {};
\draw [xshift=5cm] node[circle,fill,inner sep=8pt, color = white, label=below:$...$] {};
\draw [xshift=16cm] node[circle,fill,inner sep=3pt, color = white, label=below:\footnotesize{$241$}] {};

\draw[gray,thick,->] ({-1}, 0) -- (19, 0) {};

\tikzstyle{every node}=[draw,shape=circle]

\draw [xshift=0cm] node[circle,fill,color=green,inner sep=2pt,label=below:$1$]{};
\draw [xshift=2cm] node[circle,fill,color=green,inner sep=2pt]{};
\draw [xshift=3cm] node[circle,fill,color=green,inner sep=2pt,label=below:$16$]{};

\draw [xshift=16cm] node[circle,fill,color=red,inner sep=2pt]{};
\draw [xshift=17cm] node[circle,fill,color=red,inner sep=2pt]{};
\draw [xshift=18cm] node[circle,fill,color=red,inner sep=2pt]{};
%,label=below:$155 117 520$
\end{tikzpicture}
\end{figure}
Our result: $M < 102$.
\begin{figure}[H]
\centering
\input{./tikz/number-line-5.tikz}
\end{figure}
\end{frame}
%------------------------------------------------

%------------------------------------------------
\begin{frame}
\begin{block}{First Tool}
If the region is small enough, it doesn’t contain two points of the partial
edge drawing.
\end{block}
\begin{figure}
\centering
\resizebox{8cm}{!}{%
\begin{tikzpicture}[scale = 1.1, node distance=0.1cm,>=latex, dot/.style={circle,inner sep=1pt,fill,label={#1}, name=#1},
  extended line/.style={shorten >=-#1,shorten <=-#1},
 extended line/.default=1cm,
 dot2/.style={circle,inner sep=1.4pt,draw,fill=white,label={#1}, name=#1}]

\begin{footnotesize}
\draw [ thick] (2.44,1.6)-- (3.41,1.6);
\draw [green,ultra thick] (3.41,1.6)-- (3.31,0.7);
\draw [green,ultra thick] (3.31,0.7)-- (2.57,0.7);
\draw [green,ultra thick] (2.57,0.7)-- (2.44,1.6);

\draw [ultra thick] (3.14,1.2)-- (3.1,0.28);
\draw [ultra thick] (2.89,1.46)-- (2.91,0.47);
\draw [ultra thick] (2.89,1.46)-- (3.66,1.1);
\draw [ultra thick] (3.14,1.2)-- (3.85,0.9);
\draw [ultra thick] (2.16,1.1)-- (2.89,1.46);
\draw [ultra thick] (2.35,0.9)-- (3.14,1.2);

\draw [thick,color=blue] (2.35,0.9)-- (0,0);
\draw [thick,color=blue] (2.16,1.1)-- (0,0);
\draw [thick,color=blue] (3.1,0.28)-- (3,-2.5);
\draw [thick,color=blue] (2.91,0.47)-- (3,-2.5);
\draw [thick,color=blue] (3.85,0.9)-- (6,0);
\draw [thick,color=blue] (3.66,1.1)-- (6,0);

\node [dot=](p1) at (2.89,1.46) {};
%\node [left = of p1] {$p_{1}$};

\node [dot=](p2) at (3.14,1.2) {};
%\node [left = of p2] {$p_{2}$};

\node [dot=](y) at (6,0) {};
\node [below = of y] {$y$};

\node [dot=](z) at (3,-2.5) {};
\node [below = of z] {$z$};

\node [dot=](x) at (0,0) {};
\node [below = of x] {$x$};

\node [dot=](inf) at (0.82,-2.9) {};
\node [below = of inf] {$\infty$};


\node [dot2=](t1) at (2.44,1.6) {};
\node [dot2=](t2) at (3.41,1.6) {};
\node [dot2=](t3) at (3.31,0.7) {};
\node [dot2=](t4) at (2.57,0.7) {};

\fill[color-trapezoid, opacity=0.3] (2.44,1.6) -- (3.41,1.6) -- (3.31,0.7) -- (2.57,0.7) --  (2.44,1.6) -- cycle;
%\draw [] (t1) -- (t2) -- (t3) -- (t4) -- (t1) -- cycle;

% draw region T_0 inside T
\draw [red] plot [smooth cycle] coordinates {(2.57,1.5) (3.3,1.4) (3,0.9) };

\coordinate (t0) at (2.74, 1.18);
\node [red,below = of t0] {$Q_{0}$};

\coordinate (t) at (3.2,1.49);
\node [above = of t] {$Q$};

\coordinate (k33) at (5,-2.7);
\node [blue, right = of k33] {$K_{3,3}$};

\draw [thick,blue]  (x) to[out=-40,in=80] (inf);
\draw [thick,blue]  (y) to[out=-55,in=-20] (inf);
\draw [thick,blue]  (z) to[out=-110,in=70] (inf);

\end{footnotesize}
\end{tikzpicture}
}
\end{figure}
\end{frame}
%------------------------------------------------

%------------------------------------------------
\begin{frame}
\begin{block}{Second Tool}
If a corner region is small enough, it doesn’t contain many points of the
partial drawing.
\end{block}

\begin{figure}
\centering
\resizebox{6cm}{!}{%
\begin{tikzpicture}[scale = 6, node distance=0.1cm,>=latex, dot/.style={circle,inner sep=1pt,fill,label={#1}, name=#1},
dot2/.style={circle,inner sep=1pt,draw,fill=white,label={#1}, name=#1}]

\begin{footnotesize}
\fill[blue,opacity=0.3] (0.75,0) -- (1,0) -- (0.875, 0.216506) -- (0.7, 0.173205) -- (0.75,0) -- cycle;

\draw [ultra thin] (0.75,0) -- (1,0) -- (0.875, 0.216506) -- (0.7, 0.173205) -- (0.75,0) -- cycle;

\draw [green, ultra thick] (0.875, 0.216506) -- (0.7, 0.173205) -- (0.75,0); 

% points 
\coordinate (a) at (0,0) {};
\node [dot=](b) at (1,0) {};
\coordinate (c) at ({1/2},{sqrt(3)/2}) {};

\node [dot=] (p) at (0.77,0.15) {};
\node [dot=] (p2) at (0.86, 0.07) {};

\node [dot=] (x) at (0.625, 0.64952) {};
\node [above = of x] {$x$};

\node [dot=] (y) at (0.25, 0) {};
\node [left = of y] {$y$};

\draw[ultra thick] (p2) -- (0.77, 0.26901);
\draw[ultra thick] (p2) -- (0.64443, 0.05245);

\coordinate (c1) at (0.6875, 0.54127) {};
\coordinate (a1) at (0.375, 0) {};


% stubs between p and a
\draw [ultra thick] (p) -- (0.5775, 0.1125) {};

% stubs between p and b
\draw [ultra thick] (p) -- (0.8275, 0.1125) {};
\draw [ultra thick] (b) -- (0.9425, 0.0375) {};

% stubs betwen p2 and b
\draw[ultra thick] (p2) -- (0.9, 0.05);
\draw[ultra thick] (b) -- (0.97, 0.02);

% stubs between p and c
\draw [ultra thick] (p) -- (0.7025, 0.329006) {};

% draw stubs
\draw[ultra thick] (b) -- ({1 - 0.25*(0.5)}, {0.25*sqrt(3)/2});
\draw[ultra thick] ({0.75}, 0) -- (b);

\end{footnotesize}
\end{tikzpicture}
}
\end{figure}
\end{frame}
%------------------------------------------------

%------------------------------------------------
\begin{frame}
\frametitle{Frame of the Drawing}
\begin{figure}
\centering
\resizebox{8cm}{!}{%
\begin{tikzpicture}[scale = 7, node distance=0.1cm,>=latex, dot/.style={circle,inner sep=1.4pt,fill,label={#1}, name=#1},
dot2/.style={circle,inner sep=1.4pt,draw,fill=white,label={#1}, name=#1}]
\begin{footnotesize}
% axes
\draw[gray,thick,->] ({-0.2}, 0) -- (1.3, 0) node[right] {$x$};
\draw[gray,thick,->] (0, {-0.1545-0.2}) -- ({0, 0.464+0.3}) node[above] {$y$};

\node [dot=] at (0.1,0.2) {};
\node [dot=] at (0.1,0.3) {};
\node [dot=] at (0.2,-0.15) {};
\node [dot=] at (0.4,-0.12) {};
\node [dot=] at (0.6,0.44) {};
\node [dot=] at (0.55,0.32) {};
\node [dot=] at (0.7,-0.13) {};
\node [dot=] at (0.523,0.4) {};
\node [dot=] at (0.8,0.38) {};
\node [dot=] at (0.9,0.32) {};

\node [dot=](a) at (0,0) {};
\node [dot=](b) at (1,0) {};
\node [dot=](c) at ({1/4},0.464) {};
\node [dot=](d) at ({3/5}, -0.1545) {};
\node [dot2=](abovea) at (0,0.464) {};
\node [dot2=](aboveb) at (1,0.464) {};
\node [dot2=](undera) at (0,-0.1545) {};
\node [dot2=](underb) at (1,-0.1545) {};

\fill[color-trapezoid,opacity=0.3] (0,0.464) -- (1,0.464)  -- (1,-0.1545) -- (0,-0.1545)-- cycle;

% c
\coordinate (t) at ({1/4},0);
\draw [thin] (-0.01, 0.464) -- (0.01, 0.464);
\coordinate (cy) at (0,0.464);

% d
\draw [thin] ({1/4}, 0.01000000000) -- ({1/4}, -0.01000000000);
\coordinate (u) at (0,-0.1545);

\draw[ultra thin] (undera) -- (underb) -- (aboveb) -- (abovea) -- (undera) -- cycle;
\draw[ultra thin] (a) -- (b) -- cycle;
\node [right = of b,fill=white] {$b(1,0)$};
\node [left = of a,fill=white] {$a(0,0)$};
\node [above = of c] {$c = c(t, \cdot)$} {};
\node [below = of d] {$d$}; 

\node [below = of t] {$t$};

\node [dot2=](abovea) at (0,0.464) {};
\node [dot2=](aboveb) at (1,0.464) {};
\node [dot2=](undera) at (0,-0.1545) {};
\node [dot2=](underb) at (1,-0.1545) {};
\node [above = of abovea,fill=white] {$\abovesym{a}$};
\node [above = of aboveb] {$\abovesym{b}$};
\node [below = of undera, fill=white] {$\undersym{a}$};
\node [below = of underb] {$\undersym{b}$};
\coordinate (f) at (0.33,0.22);
\node [above = of f] {$F$};
\end{footnotesize}
\end{tikzpicture}
}
\caption{Frame $F$}
\end{figure}
\end{frame}
%------------------------------------------------

%------------------------------------------------
\begin{frame}
\frametitle{Tessellation of the Frame}
[Bruckdorfer et.\ al., 2013]:
\begin{figure}
\centering
\includegraphics[width=10cm]{./figures/bruckdorfer-tessellation.png}
\end{figure}
\end{frame}
%------------------------------------------------

%------------------------------------------------
\begin{frame}
\frametitle{Tessellation of the Frame}
Our tessellation:
\begin{figure}[!ht]
\begin{center}
\resizebox{6cm}{!}{%
\input{./tikz/our-tessellation.tikz}
}
\end{center}
\end{figure}
\end{frame}
%------------------------------------------------

%------------------------------------------------
\begin{frame}
\frametitle{The Tessellation Dependents on the Positions of Points}
\begin{figure}[!ht]
\begin{center}
%\resizebox{6cm}{!}
\end{center}
\end{figure}
\end{frame}
%------------------------------------------------

%------------------------------------------------
\begin{frame}
\frametitle{Transformation of the Drawing}
\begin{figure}[!ht]
\begin{center}
\resizebox{6cm}{!}{%
\begin{tikzpicture}[scale = 4, node distance=0.1cm,>=latex, dot/.style={circle,inner sep=1pt,fill,label={#1}, name=#1},
  extended line/.style={shorten >=-#1,shorten <=-#1},
 extended line/.default=1cm]

% S_{ab}
\draw (0,0) -- (-0.166667, -0.288675) -- (0.166667, -0.288675) -- (0.25,0) -- (0,0) -- cycle;
\fill[color-parallelogram,opacity=0.9] (0,0) -- (-0.166667, -0.288675) -- (0.166667, -0.288675) -- (0.25,0) -- (0,0) -- cycle;
% S_{ba}
\draw (1,0) -- (1.16667, -0.288675) -- (0.833333, -0.288675) -- (0.75,0) -- (1,0) -- cycle;
\fill[color-parallelogram,opacity=0.9] (1,0) -- (1.16667, -0.288675) -- (0.833333, -0.288675) -- (0.75,0) -- (1,0) -- cycle;

% S_{bc}
\draw (1,0) -- (1.33333, 0.) -- (1.16667, 0.288675) -- (0.875, 0.216506) -- (1,0) -- cycle;
\fill[color-parallelogram,opacity=0.9] (1,0) -- (1.33333, 0.) -- (1.16667, 0.288675) -- (0.875, 0.216506) -- (1,0) -- cycle;
% S_{cb}
\draw (0.5, 0.866025) -- (0.666667, 1.1547) -- (0.833333, 0.866025) -- (0.625, 0.649519) -- (1,0) -- cycle;
\fill[color-parallelogram,opacity=0.9] (0.5, 0.866025) -- (0.666667, 1.1547) -- (0.833333, 0.866025) -- (0.625, 0.649519) -- (1,0) -- cycle;

% S_{ac}
\draw (0,0) -- (-0.333333, 0.) -- (-0.166667, 0.288675) -- (0.125, 0.216506) -- (0,0) -- cycle;
\fill[color-parallelogram,opacity=0.9] (0,0) -- (-0.333333, 0.) -- (-0.166667, 0.288675) -- (0.125, 0.216506) -- (0,0) -- cycle;
% S_{ca}
\draw (0.5, 0.866025) -- (0.333333, 1.1547) -- (0.166667, 0.866025) -- (0.375, 0.649519) -- (0.5, 0.866025) -- cycle;
\fill[color-parallelogram,opacity=0.9] (0.5, 0.866025) -- (0.333333, 1.1547) -- (0.166667, 0.866025) -- (0.375, 0.649519) -- (0.5, 0.866025) -- cycle;


% color black holes
\fill[blue, opacity=0.3] ({-0.2733333333333333}, {-0.5773502691896257}) -- ({-0.00555556},{-0.5773502691896257}) -- ({0.0955556},{-0.28867513459481287})  -- ({-0.13666666666666666},{-0.28867513459481287}) --  ({-0.2733333333333333}, {-0.5773502691896257}) -- cycle;
\fill[blue, opacity=0.3] ({-0.41000000000000003},{-0.8660254037844386})  -- ({-0.137143},{-0.866025}) -- ({-0.0309524},{-0.57735}) -- ({-0.2733333333333333}, {-0.5773502691896257}) -- ({-0.41000000000000003},{-0.8660254037844386}) -- cycle;

% color trapezoids
\fill[grey, opacity=0.3] ({0.3125},{0.2165063509}) -- ({0.2500000000},{0}) -- ({0.7500000000},{0}) -- ({0.6875},{0.2165063509}) -- cycle;
\fill[grey, opacity=0.3] ({0.46875},{0.4871392896}) -- ({0.625},{0.6495190528}) -- ({0.875},{0.2165063509})-- ({0.65625},{0.1623797632}) -- cycle;
\fill[grey, opacity=0.3] ({0.53125},{0.4871392896}) -- ({0.375},{0.6495190528}) -- ({0.125},{0.2165063509}) -- ({0.34375},{0.1623797632}) -- cycle;

\fill[grey, opacity=0.3] ({0.2500000000},{0}) -- ({0.166667},{-0.2886751346}) -- ({0.833333},{-0.2886751346})--({0.7500000000},{0}) -- cycle;

\fill[grey, opacity=0.3] ({0.0955556},{-0.288675}) -- ({-0.00555556},{-0.57735}) -- ({0.954167},{-0.57735})--({0.863333},{-0.288675})-- cycle;
\fill[grey, opacity=0.3] ({-0.0309524},{-0.57735}) -- ({-0.137143},{-0.866025})-- ({0.772},{-0.866025})--({0.726667},{-0.57735}) -- cycle;
\fill[grey, opacity=0.3] ({0.375},{0.6495190528}) -- ({0.329759},{0.696535})-- ({0.0616622},{0.232178})--({0.125},{0.2165063509}) -- cycle;
\fill[grey, opacity=0.3] ({0.875},{0.2165063509}) -- ({1.16667},{0.2886751346}) -- ({0.833333},{0.8660254038})--({0.625},{0.6495190528})-- cycle;
\fill[grey, opacity=0.3] ({1.15018},{0.317225}) -- ({1.43773},{0.396532}) -- ({1.04167},{1.082531755})--({0.833333},{0.8660254038})-- cycle;

% color parallelograms
\node [dot=](a) at (0,0) {};
\node [dot=](b) at (1,0) {};
\node [dot=](c) at ({1/2},{sqrt(3)/2}) {};

%% color deltoids
% deltoid_a
\fill[blue, opacity=0.3] (0,0) -- (0.25,0) -- (0.3, 0.173205) -- (0.125, 0.216506) -- (0,0) -- cycle;
% deltoid_b
\fill[blue, opacity=0.3] (0.75,0) -- (1,0) -- (0.875, 0.216506) -- (0.7, 0.173205) -- (0.75,0) -- cycle;
% deltoid_c
\fill[blue, opacity=0.3] (0.5, 0.866025) -- (0.625, 0.649519) -- (0.5, 0.519615) -- (0.375, 0.649519) -- (0.5, 0.866025) -- cycle;

% color inside
\fill[gray, opacity=0.6] ({0.375}, {sqrt(3)/8}) -- ({0.625}, {sqrt(3)/8}) -- ({0.5}, {sqrt(3)/4}) -- cycle;
\begin{footnotesize}
\node [dot=](a) at (0,0) {};
\node [left = of a] {$a$};
\node [dot=](b) at (1,0) {};
\node [right = of b] {$b$};
\node [dot=](c) at ({1/2},{sqrt(3)/2}) {};
\node [above = of c] {$c$};

\node [dot=](overa) at ({1/2-0.09},{sqrt(3)/2}) {};
\node [above = of overa] {$\abovesym{a}$};
\node [dot=](overb) at ({1/2 + 0.91},{sqrt(3)/2}) {};

\node [dot=](undera) at ({1/2-0.91},{-sqrt(3)/2}) {};
\node [below = of undera] {$\undersym{a}$};
\node [dot=](underb) at ({1/2+0.09},{-sqrt(3)/2}) {};
\node [below = of underb] {$\undersym{b}$};

\draw[ultra thin] (undera) -- (underb) -- (overb) -- (overa) -- (undera) -- cycle;
\draw[thin,dotted] (a) -- (b) -- (c) -- (a) -- cycle;

%% big triangle 
\draw [thin,dotted] ({-1/2},{-sqrt(3)/2}) -- ({1/2},{sqrt(3)/2}) -- ({3/2},{-sqrt(3)/2}) -- cycle;
%% flipud
\draw [thin,dotted] ({-1/2},{sqrt(3)/2}) -- ({1/2},{-sqrt(3)/2}) -- ({3/2},{sqrt(3)/2}) -- cycle;

\draw [thin,dotted]({-1/3},{-sqrt(3)/3}) -- ({4/3},{-sqrt(3)/3});
\draw [thin,dotted]({-1/6},{-sqrt(3)/6}) -- ({7/6},{-sqrt(3)/6});
\draw [ thin,dotted]({-1/3},{sqrt(3)/3}) -- ({-1/6},{sqrt(3)/2});
\draw [ thin,dotted]({-1/6},{sqrt(3)/6}) -- ({1/6},{sqrt(3)/2});
\draw [ thin,dotted]({7/6},{sqrt(3)/6}) -- ({5/6},{sqrt(3)/2});
\draw [ thin,dotted]({4/3},{sqrt(3)/3}) -- ({7/6},{sqrt(3)/2});

\draw [thin,dotted] ({0.5}, {0.2165063509461096}) -- ({1/2},{-sqrt(3)/2});
\draw [ thin,dotted] ({0.5625}, {0.3247595264191645}) -- ({3/2},{sqrt(3)/2});
\draw [ thin,dotted] ({0.4375}, {0.3247595264191645}) -- ({-1/2},{sqrt(3)/2});

\draw [ultra thin] ({0.5}, {0.2165063509461096}) -- ({0.5}, {0.28867513459481287});
\draw [ultra thin] ({0.5625}, {0.3247595264191645}) -- ({0.5}, {0.28867513459481287});
\draw [ultra thin] ({0.4375}, {0.3247595264191645}) -- ({0.5}, {0.28867513459481287});

\draw[ultra thick] (b) -- ({1 - 0.25*(0.5)}, {0.25*sqrt(3)/2});
\draw[ultra thick] ({1 - 0.75*(0.5)}, {0.75*sqrt(3)/2}) -- (c);
\draw[ultra thick] (a) -- ({0.25*(0.5)}, {0.25*sqrt(3)/2});
\draw[ultra thick] ({0.75*(0.5)}, {0.75*sqrt(3)/2}) -- (c);
\draw[ultra thick] (a) -- ({0.25},0);
\draw[ultra thick] ({0.75}, 0) -- (b);


\draw [ultra thin] ({0.3125},{0.2165063509}) -- ({0.2500000000},{0});
\draw [ultra thin] ({0.375},{0.2165063509}) -- ({0.3333333333},{0});
\draw [ultra thin] ({0.458333},{0.2165063509}) -- ({0.4444444444},{0});
\draw [ultra thin] ({0.541667},{0.2165063509}) -- ({0.5555555556},{0});
\draw [ultra thin] ({0.625},{0.2165063509}) -- ({0.6666666667},{0});
\draw [ultra thin] ({0.6875},{0.2165063509}) -- ({0.7500000000},{0});
\draw [ultra thin] ({0.3125},{0.2165063509}) -- ({0.6875},{0.2165063509});

\draw [ultra thin] ({0.46875},{0.4871392896}) -- ({0.625},{0.6495190528});
\draw [ultra thin] ({0.5},{0.4330127019}) -- ({0.666667},{0.5773502692});
\draw [ultra thin] ({0.541667},{0.3608439182}) -- ({0.722222},{0.4811252243});
\draw [ultra thin] ({0.583333},{0.2886751346}) -- ({0.777778},{0.3849001795});
\draw [ultra thin] ({0.625},{0.2165063509}) -- ({0.833333},{0.2886751346});
\draw [ultra thin] ({0.65625},{0.1623797632}) -- ({0.875},{0.2165063509});
\draw [ultra thin] ({0.46875},{0.4871392896}) -- ({0.65625},{0.1623797632});

\draw [ultra thin] ({0.53125},{0.4871392896}) -- ({0.375},{0.6495190528});
\draw [ultra thin] ({0.5},{0.4330127019}) -- ({0.333333},{0.5773502692});
\draw [ultra thin] ({0.458333},{0.3608439182}) -- ({0.277778},{0.4811252243});
\draw [ultra thin] ({0.416667},{0.2886751346}) -- ({0.222222},{0.3849001795});
\draw [ultra thin] ({0.375},{0.2165063509}) -- ({0.166667},{0.2886751346});
\draw [ultra thin] ({0.34375},{0.1623797632}) -- ({0.125},{0.2165063509});
\draw [ultra thin] ({0.53125},{0.4871392896}) -- ({0.34375},{0.1623797632});

\draw[] ({-0.2733333333333333}, {-0.5773502691896257}) -- ({-0.00555556},{-0.5773502691896257}) -- ({0.0955556},{-0.28867513459481287})  -- ({-0.13666666666666666},{-0.28867513459481287}) --  ({-0.2733333333333333}, {-0.5773502691896257}) -- cycle;
% BH2
\draw[] ({-0.41000000000000003},{-0.8660254037844386})  -- ({-0.137143},{-0.866025}) -- ({-0.0309524},{-0.57735}) -- ({-0.2733333333333333}, {-0.5773502691896257}) -- ({-0.41000000000000003},{-0.8660254037844386}) -- cycle;

\draw [ultra thin] ({0.2500000000},{0}) -- ({0.166667},{-0.2886751346});
\draw [ultra thin] ({0.3125000000},{0}) -- ({0.25},{-0.2886751346});
\draw [ultra thin] ({0.3906250000},{0}) -- ({0.354167},{-0.2886751346});
\draw [ultra thin] ({0.4882812500},{0}) -- ({0.484375},{-0.2886751346});
\draw [ultra thin] ({0.5117187500},{0}) -- ({0.515625},{-0.2886751346});
\draw [ultra thin] ({0.6093750000},{0}) -- ({0.645833},{-0.2886751346});
\draw [ultra thin] ({0.6875000000},{0}) -- ({0.75},{-0.2886751346});
\draw [ultra thin] ({0.7500000000},{0}) -- ({0.833333},{-0.2886751346});
\draw [ultra thin] ({0.2500000000},{0}) -- ({0.7500000000},{0});

\draw [ultra thin] ({0.0955556},{-0.288675}) -- ({-0.00555556},{-0.57735});
\draw [ultra thin] ({0.148},{-0.288675}) -- ({0.06},{-0.57735});
\draw [ultra thin] ({0.210933},{-0.288675}) -- ({0.138667},{-0.57735});
\draw [ultra thin] ({0.286453},{-0.288675}) -- ({0.233067},{-0.57735});
\draw [ultra thin] ({0.377077},{-0.288675}) -- ({0.346347},{-0.57735});
\draw [ultra thin] ({0.485826},{-0.288675}) -- ({0.482283},{-0.57735});
\draw [ultra thin] ({0.5993},{-0.288675}) -- ({0.624124},{-0.57735});
\draw [ultra thin] ({0.693861},{-0.288675}) -- ({0.742326},{-0.57735});
\draw [ultra thin] ({0.772662},{-0.288675}) -- ({0.840827},{-0.57735});
\draw [ultra thin] ({0.838329},{-0.288675}) -- ({0.922912},{-0.57735});
\draw [ultra thin] ({0.863333},{-0.288675}) -- ({0.954167},{-0.57735});

\draw [ultra thin] ({0.0955556},{-0.288675}) -- ({0.863333},{-0.288675});
\draw [ultra thin] ({-0.00555556},{-0.57735}) -- ({0.954167},{-0.57735}); % spodnja crta

\draw [ultra thin] ({-0.0309524},{-0.57735}) -- ({-0.137143},{-0.866025});
\draw [ultra thin] ({0.0194444},{-0.57735}) -- ({-0.0766667},{-0.866025});
\draw [ultra thin] ({0.0782407},{-0.57735}) -- ({-0.00611111},{-0.866025});
\draw [ultra thin] ({0.146836},{-0.57735}) -- ({0.0762037},{-0.866025});
\draw [ultra thin] ({0.226865},{-0.57735}) -- ({0.172238},{-0.866025});
\draw [ultra thin] ({0.320231},{-0.57735}) -- ({0.284277},{-0.866025});
\draw [ultra thin] ({0.429158},{-0.57735}) -- ({0.41499},{-0.866025});
\draw [ultra thin] ({0.55624},{-0.57735}) -- ({0.567488},{-0.866025});
\draw [ultra thin] ({0.667254},{-0.57735}) -- ({0.700704},{-0.866025});
\draw [ultra thin] ({0.726667},{-0.57735}) -- ({0.772},{-0.866025});
\draw [ultra thin] ({-0.0309524},{-0.57735}) -- ({0.726667},{-0.57735});
\draw [ultra thin] ({-0.137143},{-0.866025}) -- ({0.772},{-0.866025}); 

\draw [ultra thin] ({0.375},{0.6495190528}) -- ({0.329759},{0.696535});
\draw [ultra thin] ({0.336146},{0.582222}) -- ({0.288092},{0.624366});
\draw [ultra thin] ({0.285214},{0.494006}) -- ({0.233474},{0.529765});
\draw [ultra thin] ({0.25},{0.4330127019}) -- ({0.19571},{0.464357});
\draw [ultra thin] ({0.214786},{0.372019}) -- ({0.157947},{0.398948});
\draw [ultra thin] ({0.163854},{0.283804}) -- ({0.103329},{0.304347});
\draw [ultra thin] ({0.125},{0.2165063509}) -- ({0.0616622},{0.232178});
\draw [ultra thin] ({0.375},{0.6495190528}) -- ({0.125},{0.2165063509});
\draw [ultra thin] ({0.329759},{0.696535}) -- ({0.0616622},{0.232178}); 

\draw [ultra thin] ({0.875},{0.2165063509}) -- ({1.16667},{0.2886751346});
\draw [ultra thin] ({0.84375},{0.2706329387}) -- ({1.125},{0.3608439182});
\draw [ultra thin] ({0.804688},{0.3382911734}) -- ({1.07292},{0.4510548978});
\draw [ultra thin] ({0.755859},{0.4228639667}) -- ({1.00781},{0.5638186223});
\draw [ultra thin] ({0.744141},{0.4431614371}) -- ({0.992188},{0.5908819161});
\draw [ultra thin] ({0.695313},{0.5277342304}) -- ({0.927083},{0.7036456406});
\draw [ultra thin] ({0.65625},{0.5953924651}) -- ({0.875},{0.7938566201});
\draw [ultra thin] ({0.625},{0.6495190528}) -- ({0.833333},{0.8660254038});
\draw [ultra thin] ({0.875},{0.2165063509}) -- ({0.625},{0.6495190528});
\draw [ultra thin] ({1.16667},{0.2886751346}) -- ({0.833333},{0.8660254038}); 

\draw [ultra thin] ({1.15018},{0.317225}) -- ({1.43773},{0.396532});
\draw [ultra thin] ({1.11355},{0.380671}) -- ({1.39194},{0.475838});
\draw [ultra thin] ({1.0696},{0.456805}) -- ({1.337},{0.571006});
\draw [ultra thin] ({1.01685},{0.548166}) -- ({1.27106},{0.685207});
\draw [ultra thin] ({0.958486},{0.649255}) -- ({1.19811},{0.811568});
\draw [ultra thin] ({0.909849},{0.733496}) -- ({1.13731},{0.91687});
\draw [ultra thin] ({0.869319},{0.803696}) -- ({1.08665},{1.00462});
\draw [ultra thin] ({0.835544},{0.862197}) -- ({1.04443},{1.07775});
\draw [ultra thin] ({0.833333},{0.8660254038}) -- ({1.04167},{1.082531755});
\draw [ultra thin] ({1.15018},{0.317225}) -- ({0.833333},{0.8660254038});
\draw [ultra thin] ({1.43773},{0.396532}) -- ({1.04167},{1.082531755}); 

\draw [ultra thin] ({1.30037},{0.634451}) -- ({1.49022},{0.727081});
\draw [ultra thin] ({1.23644},{0.745175}) -- ({1.41696},{0.853971});
\draw [ultra thin] ({1.17654},{0.84892}) -- ({1.34832},{0.972862});
\draw [ultra thin] ({1.16667},{0.8660254038}) -- ({1.337},{0.992465});
\draw [ultra thin] ({1.30037},{0.634451}) -- ({1.16667},{0.8660254038});
\draw [ultra thin] ({1.49022},{0.727081}) -- ({1.337},{0.992465}); 

\node [above = of overb] {$\abovesym{b}$};

\fill[grey, opacity=0.3] ({1.30037},{0.634451}) -- ({1.49022},{0.727081}) -- ({1.337},{0.992465})-- ({1.16667},{0.8660254038})-- cycle;

% idea of coverage with respect to abd in the upper quadrilateral
%\draw [blue]({-1/3},{sqrt(3)/3}) -- ({4/3},{sqrt(3)/3});

\end{footnotesize}
\end{tikzpicture}

}
\end{center}
\end{figure}
\end{frame}
%------------------------------------------------

%------------------------------------------------
\begin{frame}
\frametitle{The Tessellation Dependents on the Positions of Points}
\begin{figure}[!ht]
\begin{center}
%\resizebox{6cm}{!}{%
\textbf{\begin{subfigure}[t]{0.27\textwidth}
\centering
\resizebox{3cm}{!}{%
\input{./tikz/tessellation-left-transformation.tikz}%	
}%
\caption{$t = \frac{1}{11}, M < 102$.}
\end{subfigure}\qquad%
\begin{subfigure}[t]{0.27\textwidth}
\centering
\resizebox{2.51cm}{!}{%
\begin{tikzpicture}[scale = 4, node distance=0.1cm,>=latex, dot/.style={circle,inner sep=1.4pt,fill,label={#1}, name=#1},
dot2/.style={circle,inner sep=1.4pt,draw,fill=white,label={#1}, name=#1}]
% S_{ab}
\draw (0,0) -- (-0.166667, -0.288675) -- (0.166667, -0.288675) -- (0.25,0) -- (0,0) -- cycle;
\fill[color-parallelogram,opacity=0.9] (0,0) -- (-0.166667, -0.288675) -- (0.166667, -0.288675) -- (0.25,0) -- (0,0) -- cycle;
% S_{ba}
\draw (1,0) -- (1.16667, -0.288675) -- (0.833333, -0.288675) -- (0.75,0) -- (1,0) -- cycle;
\fill[color-parallelogram,opacity=0.9] (1,0) -- (1.16667, -0.288675) -- (0.833333, -0.288675) -- (0.75,0) -- (1,0) -- cycle;

% S_{bc}
\draw (1,0) -- (1.33333, 0.) -- (1.16667, 0.288675) -- (0.875, 0.216506) -- (1,0) -- cycle;
\fill[color-parallelogram,opacity=0.9] (1,0) -- (1.33333, 0.) -- (1.16667, 0.288675) -- (0.875, 0.216506) -- (1,0) -- cycle;
% S_{cb}
\draw (0.5, 0.866025) -- (0.666667, 1.1547) -- (0.833333, 0.866025) -- (0.625, 0.649519) -- (1,0) -- cycle;
\fill[color-parallelogram,opacity=0.9] (0.5, 0.866025) -- (0.666667, 1.1547) -- (0.833333, 0.866025) -- (0.625, 0.649519) -- (1,0) -- cycle;

% S_{ac}
\draw (0,0) -- (-0.333333, 0.) -- (-0.166667, 0.288675) -- (0.125, 0.216506) -- (0,0) -- cycle;
\fill[color-parallelogram,opacity=0.9] (0,0) -- (-0.333333, 0.) -- (-0.166667, 0.288675) -- (0.125, 0.216506) -- (0,0) -- cycle;
% S_{ca}
\draw (0.5, 0.866025) -- (0.333333, 1.1547) -- (0.166667, 0.866025) -- (0.375, 0.649519) -- (0.5, 0.866025) -- cycle;
\fill[color-parallelogram,opacity=0.9] (0.5, 0.866025) -- (0.333333, 1.1547) -- (0.166667, 0.866025) -- (0.375, 0.649519) -- (0.5, 0.866025) -- cycle;

% color trapezoids
\fill[grey, opacity=0.3] ({0.3125},{0.2165063509}) -- ({0.2500000000},{0}) -- ({0.7500000000},{0}) -- ({0.6875},{0.2165063509}) -- cycle;
\fill[grey, opacity=0.3] ({0.46875},{0.4871392896}) -- ({0.625},{0.6495190528}) -- ({0.875},{0.2165063509})-- ({0.65625},{0.1623797632}) -- cycle;
\fill[grey, opacity=0.3] ({0.53125},{0.4871392896}) -- ({0.375},{0.6495190528}) -- ({0.125},{0.2165063509}) -- ({0.34375},{0.1623797632}) -- cycle;

\fill[grey, opacity=0.3] ({0.2500000000},{0}) -- ({0.166667},{-0.2886751346}) -- ({0.833333},{-0.2886751346})--({0.7500000000},{0}) -- cycle;
\fill[grey, opacity=0.3] ({0.875},{0.2165063509}) -- ({1.16667},{0.2886751346}) -- ({0.833333},{0.8660254038})--({0.625},{0.6495190528})-- cycle;


% color deltoids
\coordinate (astar) at (0.218, 0.173205) {};
\coordinate (bstar) at (0.618, 0.173205) {};
\coordinate (cstar) at (0.254, 0.519615) {};
\node [dot=](a) at (0,0) {};
\node [dot=](b) at (1,0) {};
\node [dot=](c) at ({0.5},{sqrt(3)/2}) {};

%% color deltoids 2
% deltoid_a
\fill[blue, opacity=0.3] (0,0) -- (0.25,0) -- (0.3, 0.173205) -- (0.125, 0.216506) -- (0,0) -- cycle;
% deltoid_b
\fill[blue, opacity=0.3] (0.75,0) -- (1,0) -- (0.875, 0.216506) -- (0.7, 0.173205) -- (0.75,0) -- cycle;
% deltoid_c
\fill[blue, opacity=0.3] (0.5, 0.866025) -- (0.625, 0.649519) -- (0.5, 0.519615) -- (0.375, 0.649519) -- (0.5, 0.866025) -- cycle;

% color inside
\fill[grey, opacity=0.6] ({0.375}, {sqrt(3)/8}) -- ({0.625}, {sqrt(3)/8}) -- ({0.5}, {sqrt(3)/4}) -- cycle;
\begin{footnotesize}
\node [dot=](a) at (0,0) {};
\node [left = of a] {$a$};
\node [dot=](b) at (1,0) {};
\node [right = of b] {$b$};
\node [dot=](c) at ({1/2},{sqrt(3)/2}) {};
\node [above = of c] {$c$};

% draw the stubs
\draw[ultra thick] (b) -- ({1 - 0.25*(0.5)}, {0.25*sqrt(3)/2});
\draw[ultra thick] ({1 - 0.75*(0.5)}, {0.75*sqrt(3)/2}) -- (c);
\draw[ultra thick] (a) -- ({0.25*(0.5)}, {0.25*sqrt(3)/2});
\draw[ultra thick] ({0.75*(0.5)}, {0.75*sqrt(3)/2}) -- (c);
\draw[ultra thick] (a) -- ({0.25},0);
\draw[ultra thick] ({0.75}, 0) -- (b);
% symmetrically inside
\draw [ultra thin] ({0.3125},{0.2165063509}) -- ({0.2500000000},{0});
\draw [ultra thin] ({0.375},{0.2165063509}) -- ({0.3333333333},{0});
\draw [ultra thin] ({0.458333},{0.2165063509}) -- ({0.4444444444},{0});
\draw [ultra thin] ({0.541667},{0.2165063509}) -- ({0.5555555556},{0});
\draw [ultra thin] ({0.625},{0.2165063509}) -- ({0.6666666667},{0});
\draw [ultra thin] ({0.6875},{0.2165063509}) -- ({0.7500000000},{0});
\draw [ultra thin] ({0.3125},{0.2165063509}) -- ({0.6875},{0.2165063509});

\draw [ultra thin] ({0.46875},{0.4871392896}) -- ({0.625},{0.6495190528});
\draw [ultra thin] ({0.5},{0.4330127019}) -- ({0.666667},{0.5773502692});
\draw [ultra thin] ({0.541667},{0.3608439182}) -- ({0.722222},{0.4811252243});
\draw [ultra thin] ({0.583333},{0.2886751346}) -- ({0.777778},{0.3849001795});
\draw [ultra thin] ({0.625},{0.2165063509}) -- ({0.833333},{0.2886751346});
\draw [ultra thin] ({0.65625},{0.1623797632}) -- ({0.875},{0.2165063509});
\draw [ultra thin] ({0.46875},{0.4871392896}) -- ({0.65625},{0.1623797632});

\draw [ultra thin] ({0.53125},{0.4871392896}) -- ({0.375},{0.6495190528});
\draw [ultra thin] ({0.5},{0.4330127019}) -- ({0.333333},{0.5773502692});
\draw [ultra thin] ({0.458333},{0.3608439182}) -- ({0.277778},{0.4811252243});
\draw [ultra thin] ({0.416667},{0.2886751346}) -- ({0.222222},{0.3849001795});
\draw [ultra thin] ({0.375},{0.2165063509}) -- ({0.166667},{0.2886751346});
\draw [ultra thin] ({0.34375},{0.1623797632}) -- ({0.125},{0.2165063509});
\draw [ultra thin] ({0.53125},{0.4871392896}) -- ({0.34375},{0.1623797632});

\draw[thin,dotted] (a) -- (b) -- (c) -- (a) -- cycle;
\coordinate (undera) at (0,{-sqrt(3)/2}) {};
\coordinate (underb) at (1,{-sqrt(3)/2}) {};
\coordinate (aboveb) at (1,{sqrt(3)/2}) {};
\coordinate (abovea) at (0,{sqrt(3)/2}) {};
\draw[] (undera) -- (underb) -- (aboveb) -- (abovea) -- cycle;

% Bottom Layers
\draw [ultra thin] ({0.2500000000},{0}) -- ({0.166667},{-0.2886751346});
\draw [ultra thin] ({0.3125000000},{0}) -- ({0.25},{-0.2886751346});
\draw [ultra thin] ({0.3906250000},{0}) -- ({0.354167},{-0.2886751346});
\draw [ultra thin] ({0.4882812500},{0}) -- ({0.484375},{-0.2886751346});
\draw [ultra thin] ({0.5117187500},{0}) -- ({0.515625},{-0.2886751346});
\draw [ultra thin] ({0.6093750000},{0}) -- ({0.645833},{-0.2886751346});
\draw [ultra thin] ({0.6875000000},{0}) -- ({0.75},{-0.2886751346});
\draw [ultra thin] ({0.7500000000},{0}) -- ({0.833333},{-0.2886751346});
\draw [ultra thin] ({0.2500000000},{0}) -- ({0.7500000000},{0});

\draw [ultra thin] ({0.166667},{-0.2886751346}) --({0.0833333},{-0.5773502692});
\draw [ultra thin] ({0.233333},{-0.2886751346}) --({0.166667},{-0.5773502692});
\draw [ultra thin] ({0.313333},{-0.2886751346}) --({0.266667},{-0.5773502692});
\draw [ultra thin] ({0.409333},{-0.2886751346}) --({0.386667},{-0.5773502692});
\draw [ultra thin] ({0.5},{-0.2886751346}) -- ({0.5},{-0.5773502692});
\draw [ultra thin] ({0.590667},{-0.2886751346}) --({0.613333},{-0.5773502692});
\draw [ultra thin] ({0.686667},{-0.2886751346}) --({0.733333},{-0.5773502692});
\draw [ultra thin] ({0.766667},{-0.2886751346}) --({0.833333},{-0.5773502692});
\draw [ultra thin] ({0.833333},{-0.2886751346}) --({0.916667},{-0.5773502692});
\draw [ultra thin] ({0.166667},{-0.2886751346}) --({0.833333},{-0.2886751346});

% inside deltoids
\draw [ultra thin] ({0.5}, {0.2165063509461096}) -- ({0.5}, {0.28867513459481287});
\draw [ultra thin] ({0.5625}, {0.3247595264191645}) -- ({0.5}, {0.28867513459481287});
\draw [ultra thin] ({0.4375}, {0.3247595264191645}) -- ({0.5}, {0.28867513459481287});

\fill[grey, opacity=0.3] ({0.166667},{-0.2886751346}) --({0.0833333},{-0.5773502692})--  ({0.916667},{-0.5773502692}) --({0.833333},{-0.2886751346})  -- cycle;

\fill[blue, opacity=0.3] ({0.166667},{-0.2886751346}) --({0.0833333},{-0.5773502692}) --  ({0},{-2/3*sqrt(3)/2}) --({0},{-1/3*sqrt(3)/2})  -- cycle;
\fill[blue, opacity=0.3] ({0.833333},{-0.2886751346}) --({0.916667},{-0.5773502692}) --  ({1},{-2/3*sqrt(3)/2}) --({1},{-1/3*sqrt(3)/2})  -- cycle;

\draw [ultra thin] ({0},{-0.5773502692}) -- ({-0.1},{-0.8660254038});
\draw [ultra thin] ({0.0555556},{-0.5773502692}) -- ({-0.0333333},{-0.8660254038});
\draw [ultra thin] ({0.12037},{-0.5773502692}) -- ({0.0444444},{-0.8660254038});
\draw [ultra thin] ({0.195988},{-0.5773502692}) -- ({0.135185},{-0.8660254038});
\draw [ultra thin] ({0.284208},{-0.5773502692}) -- ({0.241049},{-0.8660254038});
\draw [ultra thin] ({0.387131},{-0.5773502692}) -- ({0.364558},{-0.8660254038});
\draw [ultra thin] ({0.5},{-0.5773502692}) -- ({0.5},{-0.8660254038});
\draw [ultra thin] ({0.612869},{-0.5773502692}) -- ({0.635442},{-0.8660254038});
\draw [ultra thin] ({0.715792},{-0.5773502692}) -- ({0.758951},{-0.8660254038});
\draw [ultra thin] ({0.804012},{-0.5773502692}) -- ({0.864815},{-0.8660254038});
\draw [ultra thin] ({0.87963},{-0.5773502692}) -- ({0.955556},{-0.8660254038});
\draw [ultra thin] ({0.944444},{-0.5773502692}) -- ({1.03333},{-0.8660254038});
\draw [ultra thin] ({1.},{-0.5773502692}) -- ({1.1},{-0.8660254038});
\draw [ultra thin] ({0},{-0.5773502692}) -- ({1.},{-0.5773502692});
\draw  [ultra thin] ({-0.1},{-0.8660254038}) -- ({1.1},{-0.8660254038}); % bottom line
\fill[grey, opacity=0.3] ({0},{-0.5773502692}) -- ({-0.1},{-0.8660254038}) --  ({1.1},{-0.8660254038}) -- ({1.},{-0.5773502692})   -- cycle;


\draw [ultra thin] ({0.875},{0.2165063509}) -- ({1.16667},{0.2886751346});
\draw [ultra thin] ({0.84375},{0.2706329387}) -- ({1.125},{0.3608439182});
\draw [ultra thin] ({0.804688},{0.3382911734}) -- ({1.07292},{0.4510548978});
\draw [ultra thin] ({0.755859},{0.4228639667}) -- ({1.00781},{0.5638186223});
\draw [ultra thin] ({0.744141},{0.4431614371}) -- ({0.992188},{0.5908819161});
\draw [ultra thin] ({0.695313},{0.5277342304}) -- ({0.927083},{0.7036456406});
\draw [ultra thin] ({0.65625},{0.5953924651}) -- ({0.875},{0.7938566201});
\draw [ultra thin] ({0.625},{0.6495190528}) -- ({0.833333},{0.8660254038});
\draw [ultra thin] ({0.875},{0.2165063509}) -- ({0.625},{0.6495190528});
\draw [ultra thin] ({1.16667},{0.2886751346}) -- ({0.833333},{0.8660254038}); % bottom line

\draw [ultra thin] ({0.125},{0.2165063509}) -- ({-0.166667},{0.2886751346});
\draw [ultra thin] ({0.15625},{0.2706329387}) -- ({-0.125},{0.3608439182});
\draw [ultra thin] ({0.195313},{0.3382911734}) -- ({-0.0729167},{0.4510548978});
\draw [ultra thin] ({0.244141},{0.4228639667}) -- ({-0.0078125},{0.5638186223});
\draw [ultra thin] ({0.255859},{0.4431614371}) -- ({0.0078125},{0.5908819161});
\draw [ultra thin] ({0.304688},{0.5277342304}) -- ({0.0729167},{0.7036456406});
\draw [ultra thin] ({0.34375},{0.5953924651}) -- ({0.125},{0.7938566201});
\draw [ultra thin] ({0.375},{0.6495190528}) -- ({0.166667},{0.8660254038});
\draw [ultra thin] ({0.125},{0.2165063509}) -- ({0.375},{0.6495190528});
\draw [ultra thin] ({-0.166667},{0.2886751346}) -- ({0.166667},{0.8660254038}); % bottom line
\fill[grey, opacity=0.3] ({0.125},{0.2165063509}) -- ({-0.166667},{0.2886751346}) -- ({0.166667},{0.8660254038}) -- ({0.375},{0.6495190528})  -- cycle;

\draw [ultra thin] ({0.166667},{0.8660254038}) -- ({0.0625},{0.9742785793});
\draw [ultra thin] ({0.129167},{0.8010734985}) -- ({0.0203125},{0.9012076858});
\draw [ultra thin] ({0.0269557},{0.6240389617}) -- ({-0.0946748},{0.7020438319});
\draw [ultra thin] ({0.0833333},{0.7216878365}) -- ({-0.03125},{0.8118988160});
\draw [ultra thin] ({0.166667},{0.8660254038}) -- ({0.0833333},{0.7216878365});
\draw [ultra thin] ({0.0833333},{0.7217}) -- ({0.0833333},{0.7216878365});
\draw [ultra thin] (0., 0.5774 ) -- (-0.125, 0.6495);
\draw [ultra thin] ({0.0625},{0.9742785793}) -- (-0.125, 0.6495);
\fill[grey, opacity=0.3] ({0.166667},{0.8660254038}) -- ({0.0625},{0.9742785793}) -- (-0.125, 0.6495) -- (0., 0.5774 ) -- cycle;

\draw [ultra thin] ({0.833333},{0.8660254038}) -- ({0.9375},{0.9742785793});
\draw [ultra thin] ({0.870833},{0.8010734985}) -- ({0.979688},{0.9012076858});
\draw [ultra thin] ({0.973044},{0.6240389617}) -- ({1.09467},{0.7020438319});
\draw [ultra thin] ({0.916667},{0.7216878365}) -- ({1.03125},{0.8118988160});
\draw [ultra thin] ({0.833333},{0.8660254038}) -- ({0.916667},{0.7216878365});
\draw [ultra thin] ({1}, {0.5774} ) -- ({1+0.125}, {0.6495});
\draw [ultra thin] ({0.9375},{0.9742785793}) -- ({1+0.125}, {0.6495}); ({1+0.125}, {0.6495})

\fill[grey, opacity=0.3] ({0.833333},{0.8660254038}) -- ({0.9375},{0.9742785793}) -- ({1+0.125}, {0.6495}) -- ({1}, {0.5774} ) -- cycle;

\node [right = of b] {$b$};
\node [left = of a] {$a$};

\node [dot2=](abovea) at (0,{sqrt(3)/2}) {};
\node [above = of abovea] {$\abovesym{a}$};
\node [dot2=](aboveb) at (1,{sqrt(3)/2}) {};
\node [above = of aboveb] {$\abovesym{b}$};
\node [dot2=](undera) at (0,{-sqrt(3)/2}) {};
\node [below = of undera] {$\undersym{a}$};
\node [dot2=](underb) at (1,{-sqrt(3)/2}) {};
\node [below = of underb] {$\undersym{b}$};
\end{footnotesize}
\end{tikzpicture}%
}%
\caption{$t = \frac{1}{2}, M < 99$.}
\end{subfigure}\qquad%
\begin{subfigure}[t]{0.27\textwidth}
\centering%
\resizebox{3cm}{!}{%
\input{./tikz/tessellation-right-transformation.tikz}%
}%
\caption{$t = \frac{10}{11}, M < 102$.}
\end{subfigure}}

%}
\end{center}
\end{figure}
\end{frame}
%------------------------------------------------


%------------------------------------------------
% TREATMENT OF REGIONS
%------------------------------------------------
\begin{frame}
\frametitle{Treatment of Regions}
\begin{figure}%[!h]
\begin{center}
\resizebox{8cm}{!}{%
\input{./tikz/relationship-2.tikz}
}
\end{center}
\caption{The corner regions.}
\end{figure}
\end{frame}
%------------------------------------------------


%------------------------------------------------
\begin{frame}
\frametitle{Treatment of Regions}
\begin{figure}%[!h]
\begin{center}
\resizebox{8cm}{!}{%
\begin{tikzpicture}[scale = 7, node distance=0.1cm,>=latex, dot/.style={circle,inner sep=1pt,fill,label={#1}, name=#1},
dot2/.style={circle,inner sep=1pt,draw,fill=white,label={#1}, name=#1}]

\begin{footnotesize}
\node [dot=](x) at (0,0) {};
\node [left = of x] {$x$};
\node [dot=](y) at (1,0) {};
\node [right = of y] {$y$};
\node [dot=](z) at (0.5,-0.87) {};
\node [below = of z] {$z$};
\draw [thin,dotted] (x) -- (y);

\draw[] (-0.03,0.64) -- (1.02,0.64) -- (0.95,0.44) -- (0.04,0.44) -- (-0.03,0.64) -- cycle;

\fill[red,opacity=0.3] (-0.03,0.64) -- (1.02,0.64) -- (0.95,0.44) -- (0.04,0.44) -- (-0.03,0.64) -- cycle;

\node [dot=](p) at (0.24,0.52) {};
\node [left = of p] {$p$};
\coordinate (px) at (0.18,0.39) {};
\coordinate (py) at (0.43,0.39) {};
\coordinate (pz) at (0.3,0.17) {};

\coordinate (xp) at (0.06,0.13) {};
\coordinate (yp) at (0.81,0.13) {};
\coordinate (zp) at (0.43,-0.52) {};

\draw [thick] (p) -- (px);
\draw [thick] (p) -- (py);
\draw [thick] (p) -- (pz);

\draw [thin] (-0.199231,0.353331) -- (-0.210769,0.346669);
\draw [thin] (-0.251231,0.443331) -- (-0.262769,0.436669);
\draw [thin] (-0.366231,0.643331) -- (-0.377769,0.636669);
\coordinate (delta) at (-0.372,0.64);
\node [left = of delta] {$\delta$};
\coordinate (gamma) at (-0.257,0.44) {};
\node [left = of gamma] {$\gamma$};
\coordinate (gammatilde) at (-0.205,0.35) {};
\node [left = of gammatilde] {$\frac{3}{4}\delta$};

\coordinate (delta1) at (1.372,0.64) {};
\coordinate (gamma1) at (1.257,0.44) {};
\coordinate (gammatilde1) at (1.205,0.35) {};
\draw [thin,dotted] (delta) -- (delta1);
\draw [thin,dotted] (gamma) -- (gamma1);
\draw [thin,dotted] (gammatilde) -- (gammatilde1);

\draw [thin] (0.195,0.00666173) -- (0.195,-0.00666173); % alpha
\draw [thin] (0.8,0.00666173) -- (0.8,-0.00666173); % beta
\draw [thin] (0.287, 0.00666173) -- (0.287, -0.00666173); %alphai
\draw [thin] (0.367, 0.00666173) -- (0.367, -0.00666173); % alphaii

\coordinate (alpha) at ({0.2-0.005},0) {};
\coordinate (beta) at (0.8,0) {};
\coordinate (alphai) at ({0.29 - 0.003},0) {};
\coordinate (alphaii) at ({0.39},0) {}; % spremenimo

\coordinate (z1) at (-0.43,0.74) {};
\coordinate (z2) at (-0.065,0.74) {}; % za -0.005 tudi alpha
\coordinate (z3) at ({0.11 - 0.005},0.74) {};
\coordinate (z4) at ({0.25 + 0.005},0.74) {};
\coordinate (z5) at ({1.05 + 0.005},0.74) {};
\coordinate (z6) at (1.43,0.74) {};
\draw [thin,dotted] (z) -- (z1);
\draw [thin,dotted] (z) -- (z2);
\draw [thin,dotted] (z) -- (z3);
\draw [thin,dotted] (z) -- (z4);
\draw [thin,dotted] (z) -- (z5);
\draw [thin,dotted] (z) -- (z6);

\node [below = of alpha,fill=white] {$\alpha$};
\node [below = of beta,fill=white] {$\omega$};
\node [below = of alphaii,fill=white] {$\beta'$};
\node [below = of alphai,fill=white] {$\beta$};

\coordinate (e) at (0.178,0.44) {};
\coordinate (f) at (0.301,0.44) {};

\coordinate (g) at (0.27,0.64) {};
\coordinate (h) at (0.13,0.64) {};

\draw (e) -- (f) -- (g) -- (h) -- (e) -- cycle;
\coordinate (bandL) at (0.79, 0.54) {};
\node [below = of bandL] {$L$};

\coordinate (bandL) at (0.247, 0.64) {};
\node [below = of bandL] {$C$};
\end{footnotesize}
\end{tikzpicture}
}
\end{center}
\caption{Maximal cell $C$ in layer $L$.}
\end{figure}
\end{frame}
%------------------------------------------------

%------------------------------------------------
\begin{frame}
\begin{figure}[!h]
\begin{center}
\resizebox{9cm}{!}{%
\begin{tikzpicture}[scale = 6, node distance=0.1cm,>=latex, dot/.style={circle,inner sep=1pt,fill,label={#1}, name=#1},
dot2/.style={circle,inner sep=1pt,draw,fill=white,label={#1}, name=#1}]

\begin{footnotesize}
\draw (0,0) -- (0.125, 0.216506) -- (-0.166667, 0.288675) -- (-0.333333, 0.) -- (0,0) -- cycle;
\fill[color-parallelogram,opacity=0.9] (0,0) -- (0.125, 0.216506) -- (-0.166667, 0.288675) -- (-0.333333, 0.) -- (0,0) -- cycle;

%% S_{ca}
\draw (0.375, 0.649519) -- (0.5,0.87) -- (0.333333, 1.1547) -- (0.166667, 0.866025) -- (0.375, 0.649519) -- cycle;
\fill[color-parallelogram,opacity=0.9] (0.375, 0.649519) -- (0.5,0.87) -- (0.333333, 1.1547) -- (0.166667, 0.866025) -- (0.375, 0.649519) -- cycle;

\node [dot=](a) at (0,0) {};
\node [below = of a] {$a$};
\node [dot=](b) at (1,0) {};
\node [below = of b] {$b$};
\node [dot=](c) at (0.5,0.87) {};
\node [above = of c] {$c$};
%
\draw [thin] ({0.375},{0.6525}) -- ({0.223602},{0.810559});
\draw [thin] ({0.340833},{0.59305}) -- ({0.181159},{0.736708});
\draw [thin] ({0.297328},{0.51735}) -- ({0.127115},{0.642671});
\draw [thin] ({0.25},{0.435}) -- ({0.068323},{0.540373});
\draw [thin] ({0.202672},{0.35265}) -- ({0.00953071},{0.438074});
\draw [thin] ({0.159167},{0.27695}) -- ({-0.0445135},{0.344037});
\draw [thin] ({0.125},{0.2175}) -- ({-0.0869565},{0.270186});
\draw [thin] ({0.375},{0.6525}) -- ({0.125},{0.2175});

\draw [thin] ({-0.0869565},{0.270186}) -- ({0.223602},{0.810559});
\fill[red,opacity=0.3] ({0.375},{0.6525}) -- ({0.125},{0.2175}) -- ({-0.0869565},{0.270186}) -- ({0.223602},{0.810559}) -- cycle;

\coordinate[]  (aa) at (0.24,0.87) {};
\coordinate[]  (bb) at (1.24,0.87) {};

\coordinate[]  (b12) at (-0.4987,0.869) {};
\coordinate[]  (b14b14) at (-0.75,0.435) {};
\coordinate[]  (b34b34) at (0.04,1) {};
\coordinate[]  (a2) at (-1,0) {};
\coordinate[]  (ac) at (0.13,0.22) {};
\coordinate[]  (ca) at (0.38,0.65) {};
\coordinate[]  (a43) at (-0.33,0) {};
\coordinate[]  (a43a43) at (0.24,1) {};
\coordinate[]  (a53) at (-0.67,0) {};

\coordinate[]  (e) at (0.23,0.8) {};
\coordinate[]  (a53a53) at (-0.09,1) {};
\coordinate[]  (a2a2) at (-0.42,1) {};
\coordinate[]  (o) at (0.42,1.01) {};

\draw[ultra thick] (b) -- ({1 - 0.25*(0.5)}, {0.25*sqrt(3)/2});
\draw[ultra thick] ({1 - 0.75*(0.5)}, {0.75*sqrt(3)/2}) -- (c);
\draw[ultra thick] (a) -- ({0.25*(0.5)}, {0.25*sqrt(3)/2});
\draw[ultra thick] ({0.75*(0.5)}, {0.75*sqrt(3)/2}) -- (c);
\draw[ultra thick] (a) -- ({0.25},0);
\draw[ultra thick] ({0.75}, 0) -- (b);

\draw [thin] (a) -- (b) -- (c) -- (a) -- cycle;
\draw [thin] (a) -- (b) -- (bb) -- (aa) -- (a) -- cycle;
\draw [dotted] (a) -- (b12) -- (c) -- cycle;

\draw [dotted] (b) -- (a2);
\draw [dotted] (b) -- (b14b14);
\draw [dotted] (b) -- (b12);
\draw [dotted] (b) -- (b34b34);
\draw [dotted] (b) -- (o);

\draw [dotted] (a43) -- (a43a43);
\draw [dotted] (a53) -- (a53a53);
\draw [dotted] (a2) -- (a2a2);

\draw [thin] (-0.33,0.00666173) -- (-0.33,-0.00666173); 
\node [below = of a43] {$\frac{4}{3}$};

\draw [thin] (-0.67,0.00666173) -- (-0.67,-0.00666173); 
\node [below = of a53] {$\frac{5}{3}$};

\draw [thin] (-1,0.00666173) -- (-1,-0.00666173); 
\node [below = of a2] {$2$};

\node [dot2=] (aa) at (0.24,0.87) {};
\node [above = of aa] {$\abovesym{a}$};
\node [dot2=] (bb) at (1.24,0.87) {};
\node [above = of bb] {$\abovesym{b}$};

\node [dot2=] (e) at (0.222,0.809) {};
\node [right = of e] {$e$};

\end{footnotesize}
\end{tikzpicture}
}
\end{center}
\caption{First layer for $t = 0.24.$}
\end{figure}
\end{frame}
%------------------------------------------------

%------------------------------------------------
\begin{frame}
\begin{figure}[t]
\begin{center}
\includegraphics[width=0.8\textwidth]{./figures/plot-kL2-jL2-large.pdf}
\end{center}
\caption{Plot of the function $k_{L_{2}}(t)$ in blue and plot of the function $j_{L_{2}}(t)$ in red as functions of $t$.}
\end{figure}
\end{frame}
%------------------------------------------------

%------------------------------------------------
\begin{frame}
\begin{figure}[t]
\centering
\includegraphics[width=0.8\textwidth]{./figures/left-and-right-layers-estimates-small.pdf}
\caption{The estimate for the number of points in the left and right triangle as a function of $t$.}
\end{figure}
\end{frame}
%------------------------------------------------

%------------------------------------------------
\begin{frame}
\begin{figure}[!ht]
\begin{center}
\resizebox{4cm}{!}{%
\input{./tikz/half-shadows-1.tikz}
}
\end{center}
\caption{Illustration for very small values of the parameter $t$.}
\end{figure}
\end{frame}
%------------------------------------------------


%------------------------------------------------
\begin{frame}
\begin{figure}[!ht]
\begin{center}
\resizebox{4cm}{!}{%
\begin{tikzpicture}[scale = 5, node distance=0.1cm,>=latex, dot/.style={circle,inner sep=1pt,fill,label={#1}, name=#1},
dot2/.style={circle,inner sep=1pt,draw,fill=white,label={#1}, name=#1}]

\begin{footnotesize}

% S_{ab}
\draw (0,0) -- (-0.125, -0.33) -- (0.208333, -0.33) -- (0.25,0) -- (0,0) -- cycle;
\fill[color-parallelogram,opacity=0.9] (0,0) -- (-0.125, -0.33) -- (0.208333, -0.33) -- (0.25,0) -- (0,0) -- cycle;

% S_{ca}
\draw (1,0) -- (1.20833, -0.33) -- (0.875, -0.33) -- (0.75,0) -- (1,0) -- cycle;
\fill[color-parallelogram,opacity=0.9] (1,0) -- (1.20833, -0.33) -- (0.875, -0.33) -- (0.75,0) -- (1,0) -- cycle;

\draw[gray,thick,->] ({-.2}, 0) -- (1.2, 0) node[right] {$x$};
\draw[gray,thick,->] (0, {-1.2}) -- (0, {1.2}) node[above] {$y$};

\node[dot2=] (aabove) at (0,1) {};
\node [left = of aabove] {$\abovesym{a}$};
\node[dot2=] (abelow) at (0,-1) {};
\node [left = of abelow] {$\undersym{a}$}; %(0,-1)
\node[dot2=] (babove) at (1,1) {};
\node [right = of babove] {$\abovesym{b}$};
\node[dot2=] (bbelow) at (1,-1) {};
\node [right = of bbelow] {$\undersym{b}$};

\coordinate (ab) at (0.25,0) {};
\coordinate (ba) at (0.75,0) {};


\node[dot2=] (h) at (0,-0.33) {};
\node [left = of h] {$h$}; %(0,-\frac{1}{3})
\node[dot2=] (e) at (0,-0.66) {};
\node [left = of e] {$e$}; %(0,-\frac{2}{3})

\coordinate (ee) at ({1},{-0.66}) {};
\coordinate (hh) at (1,-0.33) {};

\coordinate (abelowb) at ({0.25+0.003},{-0.75}) {};

\node [dot=](a) at (0,0) {};
\node [left = of a,fill=white] {$a$};
\node [dot=](b) at (1,0) {};
\node [right = of b,fill=white] {$b$};

\coordinate (c38) at (0.37,1) {};
\coordinate (g38) at (0.21,-0.33) {};
\coordinate (f38) at (0.17,-0.66) {};
\coordinate (fbelow38) at  (0.13,-1) {};
\coordinate (g38g38) at (0.88,-0.33) {};

\node [dot=](c) at (0.37,1) {};
\node [above = of c,fill = white] {$c$};

\coordinate (eb) at (0.25,-0.5) {};

\coordinate (g) at (0.21,-0.33) {};
\coordinate (f) at (0.17,-0.66) {};
\coordinate (fbelow) at (0.13,-1) {};
\coordinate (fpresekokvir) at ({0.33+0.003333},{-1}) {};
\coordinate (c14presekokvir) at (0.5,-1) {};
\coordinate (c34presekokvir) at (1.5,-1) {};
\coordinate (gbelow) at (0.24,-0.66) {};

\draw (abelow) -- (fbelow) -- (f) -- (e) -- (abelow) -- cycle;
\draw (e) -- (f) -- (g) -- (h) -- (e) -- cycle;

\fill[blue,opacity=0.3] (0,-1) -- (fbelow) -- (f) -- (0, {-2/3}) -- (abelow) -- cycle;
\fill[blue,opacity=0.3] (0, {-2/3}) -- (f) -- (g) -- (0,{-1/3}) -- (e) -- cycle;

\draw (ee) -- (hh) -- (0.88,-0.33) -- (ee) -- cycle;
\fill[blue,opacity=0.3] (ee) -- (hh) -- (0.88,-0.33) -- (ee) -- cycle;

\draw[ultra thick] (a) -- (ab);
\draw[ultra thick] (b) -- (ba);

\draw [thin] (a) -- (b) -- (c) -- (a) -- cycle;
\draw [thin] (abelow) -- (bbelow) -- (babove) -- (aabove) -- (abelow) -- cycle;

\draw [dotted] (c) -- (fbelow);
\draw [dotted] (c) -- ({1+0.001},{-0.66+0.00666});

\draw [dotted] ({0},{-1/3+0.004}) -- ({1},{-1/3+0.004});
\draw [dotted] (e) -- (ee);

\node[dot2=] (h) at (0,-0.33) {};
\node[dot2=] (e) at (0,-0.66) {};
\node[dot2=] (abelow) at (0,-1) {};
\node[dot2=] (ab) at (0.25,0) {};
\node [above = of ab,fill=white] {$a_{b}$};
\node[dot2=] (ba) at (0.75,0) {};
\node [above = of ba,fill=white] {$b_{a}$};

\node[dot2=] (n) at (1,-0.33) {};
\node [right = of n] {$n$}; %(1,-\frac{1}{3})
\node[dot2=] (m) at (1,-0.66) {};
\node [right = of m] {$m$}; %(1,-\frac{2}{3})

\coordinate (p1) at (0.03,{-2/3+0.25}) {};
\node [right = of p1] {$P_{1}$};

\coordinate (p2) at (0.014,{-2/3-0.1}) {};
\node [right = of p2] {$P_{2}$};

\coordinate (p3) at ({1-0.129},{-2/3+0.25}) {};
\node [right = of p3] {$P_{3}$};

\node[dot2=] (g) at (0.21,-0.33) {};
\node[right = of g,fill=white] {$g$};
\node[dot2=]  (f) at (0.17,-0.66) {};
\node[right = of f,fill=white] {$f$};

\end{footnotesize}
\end{tikzpicture}
}
\end{center}
\caption{Illustration for $t = \frac{3}{8}$.}
\end{figure}
\end{frame}
%------------------------------------------------


%------------------------------------------------
\begin{frame}
\frametitle{Calculations}
\begin{block}{Example}
\begin{align*}
\frac{\left|pc\right|}{\left|\proj_{s} \left(\overline{p'c}\right)\right|} &\leq
\frac{\max\left\{ \left|ec\right|, \left|fc\right| \right\}}{\min
\left\{
\left|\proj_{\overline{gc}} \left(\overline{hc}\right)\right|,
\left|\proj_{\overline{hc}} \left(\overline{gc}\right)\right|
\right\}
} = q_{c}(t) = \\[6pt]%
&= \vertwopartdef 
{
\frac{
\max
\left\{ \sqrt{\frac{25}{9} + \frac{49}{576}\left(1-4t\right)^{2}}, \sqrt{\frac{25}{9}+t^{2}} \right\}
}{
\min
\left\{ \frac{128 + 15t(4t-1)}{24 \sqrt{16+9t^{2}}}, 
\frac{128 + 15t(4t-1)}{3 \sqrt{1049+200t(2t-1)}} \right\}}, } {t \leq \frac{1}{4},} 
{
\frac{
\max
\left\{ \frac{5}{3} \sqrt{1 + \left(t - \frac{1}{4}\right)^{2}}, \sqrt{\frac{25}{9}+t^{2}} \right\}
}{
\min
\left\{ \frac{16 + 3t(4t-1)}{3 \sqrt{16+9t^{2}}}, 
\frac{16 + 3t(4t-1)}{3 \sqrt{17 + 8t(2t-1)}} \right\}}, } {t \geq \frac{1}{4}.} \\[6pt]%
\end{align*}
\end{block}
\end{frame}
%------------------------------------------------


%------------------------------------------------
\begin{frame}
\begin{figure}%[!h]
\begin{center}
\resizebox{6cm}{!}{%
\begin{tikzpicture}[scale = 5, node distance=0.1cm,>=latex, dot/.style={circle,inner sep=1pt,fill,label={#1}, name=#1},
dot2/.style={circle,inner sep=1pt,draw,fill=white,label={#1}, name=#1}]

\begin{footnotesize}
% S_{ab}
\draw (0,0) -- (-0.166667, -0.288675) -- (0.166667, -0.288675) -- (0.25,0) -- (0,0) -- cycle;
\fill[color-parallelogram,opacity=0.9] (0,0) -- (-0.166667, -0.288675) -- (0.166667, -0.288675) -- (0.25,0) -- (0,0) -- cycle;

% S_{ba}
\draw (1,0) -- (1.16667, -0.288675) -- (0.833333, -0.288675) -- (0.75,0) -- (1,0) -- cycle;
\fill[color-parallelogram,opacity=0.9] (1,0) -- (1.16667, -0.288675) -- (0.833333, -0.288675) -- (0.75,0) -- (1,0) -- cycle;

\node [dot=](a) at (0,0) {};
\node [left = of a] {$a$};
\node [dot=](b) at (1,0) {};
\node [right = of b] {$b$};
\node [dot=](c) at ({1/2},{sqrt(3)/2}) {};
\node [above = of c] {$c$};

\coordinate (e) at (-0.06541, -0.57937) {};
\coordinate (h) at (-0.03315, -0.29366) {};
\coordinate (n) at (0.96821, -0.2816) {};
\coordinate (m) at (0.93514, -0.57456) {};
\coordinate (ab) at (0.25,0) {};
\coordinate (ba) at (0.75,0) {};

\node [dot2=](overa) at ({1/2-0.4},{sqrt(3)/2}) {};
\node [above = of overa] {$\abovesym{a}$};
\node [dot2=](overb) at ({1/2 + 0.6},{sqrt(3)/2}) {};

\node [dot2=](undera) at ({1/2-0.6},{-sqrt(3)/2}) {};
\node [below = of undera] {$\undersym{a}$};
\node [dot2=](underb) at ({1/2+0.4},{-sqrt(3)/2}) {};
\node [below = of underb] {$\undersym{b}$};

\draw[ultra thin] (undera) -- (underb) -- (overb) -- (overa) -- (undera) -- cycle;
\draw[thin,dotted] (a) -- (b) -- (c) -- (a) -- cycle;

\draw [thin,dotted] ({-1/2},{-sqrt(3)/2}) -- ({1/2},{sqrt(3)/2}) -- ({3/2},{-sqrt(3)/2}) -- cycle;

\draw [thin,dotted]({-1/3},{-sqrt(3)/3}) -- ({4/3},{-sqrt(3)/3});
\draw [thin,dotted]({-1/6},{-sqrt(3)/6}) -- ({7/6},{-sqrt(3)/6});

\draw [thin,dotted] (c) -- (0.00332, -0.86603);

\draw[ultra thick] (a) -- ({0.25},0);
\draw[ultra thick] ({0.75}, 0) -- (b);


\coordinate (e) at (-0.06541, -0.57937) {};
\coordinate (h) at (-0.03315, -0.29366) {};
\coordinate (n) at (0.96821, -0.2816) {};
\coordinate (m) at (0.93514, -0.57456) {};

\coordinate (f) at (0.16815, -0.29124) {};
\coordinate (g) at (0.08573, -0.57864) {};
\coordinate (ff) at (0.83, -0.289) {};
\coordinate (gg) at (0.91, -0.578) {};
\coordinate (uu) at (0.997, -0.86603) {};

% B1
\draw[] (ab) -- (ba) -- (ff) -- (f) -- (ab)-- cycle;
\fill[red,opacity=0.3] (ab) -- (ba) -- (ff) -- (f) -- (ab)-- cycle;

% B2
\draw[] (ff) -- (f) -- (g)-- (gg) -- (ff)-- cycle;
\fill[red,opacity=0.3] (ff) -- (f) -- (g)-- (gg) -- (ff)-- cycle;

% B3
\draw[] (1.02407, -0.869) -- (m)-- (e) --(-0.17754, -0.86603) --(1.02407, -0.869) -- cycle;
\fill[red,opacity=0.3] (1.02407, -0.869) -- (m)-- (e) -- (-0.17754, -0.86603) --(1.02407, -0.869) -- cycle;

\draw[thin,dotted] (c) -- (uu);

\node [dot2=](e) at (-0.06541, -0.5793) {};
\node [left = of e,fill=white] {$e$};
\node [dot2=](h) at (-0.03315, -0.29) {};
\node [left = of h] {$h$};
\node [dot2=](n) at (0.96821, -0.288) {};
\node [right = of n] {$n$};
\node [dot2=](m) at (0.93514, -0.5748) {};
\node [right = of m,fill=white] {$m$};
\node[dot2=] (ab) at (0.25,0) {};
\node [above = of ab,fill=white] {$a_{b}$};
\node[dot2=] (ba) at (0.75,0) {};
\node [above = of ba,fill=white] {$b_{a}$};
\node [above = of overb,fill=white] {$\abovesym{b}$};

%%% P1,P2,P3
\coordinate (p1) at (0.1,{-0.33-0.04}) {};
\node [left = of p1] {$P_{1}$};

\coordinate (p2) at (0.85,{-0.33-0.04}) {};
\node [right = of p2] {$P_{3}$};
% Bi
\coordinate (b1) at (0.7,{-0.12}) {};
\node [left = of b1] {$B_{1}$};
\coordinate (b2) at (0.74,{-0.33-0.07}) {};
\node [left = of b2] {$B_{2}$};
\coordinate (b3) at (0.78,{-0.66-0.04}) {};
\node [left = of b3] {$B_{3}$};

\node [dot2=](undera) at ({1/2-0.6},{-sqrt(3)/2}) {};
\node [dot2=](underb) at ({1/2+0.4},{-sqrt(3)/2}) {};

\end{footnotesize}
\end{tikzpicture}
}
\end{center}
\caption{The bottom layers for $t = 0.4$.}
\end{figure}
\end{frame}
%------------------------------------------------

%------------------------------------------------
\begin{frame}
\begin{figure}[t]
\centering
\includegraphics[width=0.8\textwidth]{./figures/lower-layer-kj.pdf}
\caption{Plot of the function $k_{B_{3}}(t)$ in blue and plot of $j_{B_{3}}(t)$ in red as functions of $t$.}
\end{figure}
\end{frame}
%------------------------------------------------

%------------------------------------------------
\begin{frame}
\begin{block}{How to find critical values of $t$}
\begin{align*}
\beta_{i} &= \frac{3\delta}{1+6\delta}, \\[6pt]%
\left(\frac{1+3\delta}{3\delta}\right)^{i}\beta_{0} &= \frac{3\delta}{1+6\delta}, \\[6pt]%
\alpha(t) &= \frac{3\delta}{1+6\delta}\left(\frac{3\delta}{1+3\delta}\right)^{i}, \\[6pt]%
t &= \overset{-1}{\alpha}\left( \frac{\left(3\delta\right)^{i+1}}{(1+6\delta)\left(1+3\delta\right)^{i}} \right).
\end{align*}	
\end{block}
\end{frame}
%------------------------------------------------

%------------------------------------------------
% CONCLUSION
%------------------------------------------------

%------------------------------------------------
\begin{frame}
\frametitle{Evaluation of the Result}
\begin{figure}[t]
\centering
\includegraphics[width=0.8\textwidth]{./figures/all-together-more-than-two-thirds.pdf}%
\caption{Plot of the estimate $M$.}
\end{figure}
\end{frame}
%------------------------------------------------


%------------------------------------------------
\begin{frame}
\frametitle{Evaluation of the Result}
\begin{figure}[t] %[b]{0.4\textwidth}
\centering
\resizebox{4cm}{!}{%
\begin{tikzpicture}[scale = 4, node distance=0.1cm,>=latex, dot/.style={circle,inner sep=1.4pt,fill,label={#1}, name=#1},
dot2/.style={circle,inner sep=1.4pt,draw,fill=white,label={#1}, name=#1}]

\draw (0,0) -- (-0.03, -0.288675) -- (0.303333, -0.288675) -- (0.25, 0.) -- (0,0) -- cycle;
\fill[color-parallelogram,opacity=0.9] (0,0) -- (-0.03, -0.288675) -- (0.303333, -0.288675) -- (0.25, 0.) -- (0,0) -- cycle;
\draw (1,0) -- (1.30333, -0.288675) -- (0.97, -0.288675) -- (0.75, 0.) -- (1,0) -- cycle;
\fill[color-parallelogram,opacity=0.9] (1,0) -- (1.30333, -0.288675) -- (0.97, -0.288675) -- (0.75, 0.) -- (1,0) -- cycle;

\draw (1,0) -- (1.33333,0) -- (1.03, 0.288675) -- (0.7725, 0.216506) -- (1,0) -- cycle;
\fill[color-parallelogram,opacity=0.9] (1,0) -- (1.33333,0) -- (1.03, 0.288675) -- (0.7725, 0.216506) -- (1,0) -- cycle;
\draw (0.09, 0.866025) -- (0.12, 1.1547) -- (0.423333, 0.866025) -- (0.3175, 0.649519) -- (0.09, 0.866025) -- cycle;
\fill[color-parallelogram,opacity=0.9] (0.09, 0.866025) -- (0.12, 1.1547) -- (0.423333, 0.866025) -- (0.3175, 0.649519) -- (0.09, 0.866025) -- cycle;

\draw (0,0) -- (-0.333333, 0.) -- (-0.303333, 0.288675) -- (0.0225, 0.216506) -- (0,0) -- cycle;
\fill[color-parallelogram,opacity=0.9] (0,0) -- (-0.333333, 0.) -- (-0.303333, 0.288675) -- (0.0225, 0.216506) -- (0,0) -- cycle;
\draw (0.09, 0.866025) -- (-0.213333, 1.1547) -- (-0.243333, 0.866025) -- (0.0675, 0.649519) -- (0.09, 0.866025) -- cycle;
\fill[color-parallelogram,opacity=0.9] (0.09, 0.866025) -- (-0.213333, 1.1547) -- (-0.243333, 0.866025) -- (0.0675, 0.649519) -- (0.09, 0.866025) -- cycle;

\coordinate (abovea) at (0,{sqrt(3)/2}) {};
\coordinate (aboveb) at (1,{sqrt(3)/2}) {};
\coordinate (undera) at (0,{-sqrt(3)/2}) {};
\coordinate (underb) at (1,{-sqrt(3)/2}) {};

\draw[ultra thin] (undera) -- (underb) -- (aboveb) -- (abovea) -- (undera) -- cycle;

\fill[grey, opacity=0.3] ({1.},{0.634451}) -- ({1.146},{0.727081}) -- ({0.86714},{0.992465}) -- ({0.756667},{0.8660254038}) -- cycle;
\fill[grey, opacity=0.3] ({1.},{0.317225}) -- ({1.25},{0.396532})  -- ({0.529167},{1.082531755}) -- ({0.423333},{0.8660254038})  -- cycle;
\fill[grey, opacity=0.3] ({0.7725},{0.2165063509}) -- ({1.03},{0.2886751346}) -- ({0.423333},{0.8660254038}) -- ({0.3175},{0.6495190528})  -- cycle;

\fill[grey, opacity=0.3] ({0.242381},{-0.57735}) -- ({0.272857},{-0.866025}) -- ({1.182},{-0.866025}) -- ({1.},{-0.57735}) -- cycle;
\fill[grey, opacity=0.3] ({0.232222},{-0.288675}) -- ({0.267778},{-0.57735}) --  ({1.2275},{-0.57735}) -- ({1.},{-0.288675}) -- cycle;
\fill[grey, opacity=0.3] ({0.7500000000},{0}) -- ({0.97},{-0.2886751346}) -- ({0.303333},{-0.2886751346})-- ({0.2500000000},{0}) -- cycle;
\fill[grey, opacity=0.3] ({0.21},{0.2165063509}) -- ({0.2500000000},{0}) --({0.7500000000},{0}) -- ({0.585},{0.2165063509}) -- cycle;
\fill[grey, opacity=0.3] ({0.238125},{0.4871392896}) -- ({0.3175},{0.6495190528}) --({0.7725},{0.2165063509}) --  ({0.579375},{0.1623797632}) -- cycle;
\fill[grey, opacity=0.3] ({0.300625},{0.4871392896}) -- ({0.0675},{0.6495190528}) --({0.0225},{0.2165063509}) --  ({0.266875},{0.1623797632}) -- cycle;

\fill[blue, opacity=0.3] ({0.272857},{-0.866025}) -- ({0.242381},{-0.57735}) -- ({0}, {-0.57735}) -- (0,{-0.866025}) -- ({0.272857},{-0.866025}) -- cycle;
\fill[blue, opacity=0.3] ({0}, {-0.57735}) -- ({0.267778},{-0.57735}) -- ({0.232222},{-0.288675}) -- ({0},{-0.288675}) -- ({0}, {-0.57735}) -- cycle;

\coordinate (astar) at (0.218, 0.173205) {};
\coordinate (bstar) at (0.618, 0.173205) {};
\coordinate (cstar) at (0.254, 0.519615) {};
\node [dot=](a) at (0,0) {};
\node [dot=](b) at (1,0) {};
\node [dot=](c) at ({0.09},{sqrt(3)/2}) {};

\fill[blue, opacity=0.3] (a) -- ({0 - 0.25*(0 - 0.09)}, {0 + 0.25*(sqrt(3)/2-0)}) -- (astar) -- ({0.25},0) -- (a) -- cycle;
\fill[blue, opacity=0.3] (b) -- ({1 - 0.25*(1 - 0.09)}, {0 + 0.25*(sqrt(3)/2-0)}) --  (bstar) -- ({0.75}, 0) -- (b) -- cycle;
\fill[blue, opacity=0.3] (c) -- ({1 - 0.75*(1 - 0.09)}, {0 + 0.75*(sqrt(3)/2-0)}) --  (cstar) -- ({0 - 0.75*(0 - 0.09)}, {0 + 0.75*(sqrt(3)/2-0)}) -- (c) -- cycle;

\coordinate[] (x) at ({0.2725},{sqrt(3)/8});
\coordinate[] (y) at ({.5225},{sqrt(3)/8});
\coordinate[] (z) at ({0.29500000000000004},{2*sqrt(3)/8});
\fill[grey, opacity=0.6] (x) -- (y) -- (z) -- cycle;
\begin{footnotesize}
\node [dot=](a) at (0,0) {};
\node [dot=](b) at (1,0) {};

\node [dot2=](abovea) at (0,{sqrt(3)/2}) {};
\node [dot2=](aboveb) at (1,{sqrt(3)/2}) {};
\node [dot2=](undera) at (0,{-sqrt(3)/2}) {};
\node [dot2=](underb) at (1,{-sqrt(3)/2}) {};

\node [dot=](c) at ({0.09},{sqrt(3)/2}) {};
\node [above = of c] {$c$} {};
\draw[ultra thin] (a) -- (b) -- (c) -- (a) -- cycle;

\draw[ultra thick] (b) -- ({1 - 0.25*(1 - 0.09)}, {0 + 0.25*(sqrt(3)/2-0)});
\draw[ultra thick] ({1 - 0.75*(1 - 0.09)}, {0 + 0.75*(sqrt(3)/2-0)}) -- (c);
\draw[ultra thick] (a) -- ({0 - 0.25*(0 - 0.09)}, {0 + 0.25*(sqrt(3)/2-0)});
\draw[ultra thick] ({0 - 0.75*(0 - 0.09)}, {0 + 0.75*(sqrt(3)/2-0)}) -- (c);
\draw[ultra thick] (a) -- ({0.25},0);
\draw[ultra thick] ({0.75}, 0) -- (b);


\draw[] ({0}, {-0.57735}) -- ({0.267778},{-0.57735}) -- ({0.232222},{-0.288675}) -- ({0},{-0.288675}) -- ({0}, {-0.57735}) -- cycle;
\draw[] ({0.272857},{-0.866025}) -- ({0.242381},{-0.57735}) -- ({0}, {-0.57735}) -- (0,{-0.866025}) -- ({0.272857},{-0.866025}) -- cycle;


\draw [ultra thin] ({0.21},{0.2165063509}) -- ({0.2500000000},{0});
\draw [ultra thin] ({0.585},{0.2165063509}) -- ({0.7500000000},{0});
\draw [ultra thin] ({0.21},{0.2165063509}) -- ({0.585},{0.2165063509});
\draw [ultra thin] ({0.238125},{0.4871392896}) -- ({0.3175},{0.6495190528});
\draw [ultra thin] ({0.579375},{0.1623797632}) -- ({0.7725},{0.2165063509});
\draw [ultra thin] ({0.238125},{0.4871392896}) -- ({0.579375},{0.1623797632});
\draw [ultra thin] ({0.300625},{0.4871392896}) -- ({0.0675},{0.6495190528});
\draw [ultra thin] ({0.266875},{0.1623797632}) -- ({0.0225},{0.2165063509});
\draw [ultra thin] ({0.300625},{0.4871392896}) -- ({0.266875},{0.1623797632});

 \draw [ultra thin] ({0.2500000000},{0}) -- ({0.303333},{-0.2886751346}); 
 \draw [ultra thin] ({0.7500000000},{0}) -- ({0.97},{-0.2886751346}); 
 \draw [ultra thin] ({0.2500000000},{0}) -- ({0.7500000000},{0}); 


 \draw [ultra thin] ({0.232222},{-0.288675}) -- ({0.267778},{-0.57735});  
 \draw [ultra thin] ({1.},{-0.288675}) -- ({1.2275},{-0.57735}); 
 \draw [ultra thin] ({0.232222},{-0.288675}) -- ({1.},{-0.288675}); 
 \draw [ultra thin] ({0.267778},{-0.57735})  -- ({1.2275},{-0.57735}); 


 \draw [ultra thin] ({0.242381},{-0.57735}) -- ({0.272857},{-0.866025}); 
 \draw [ultra thin] ({1.},{-0.57735}) -- ({1.182},{-0.866025}); 
 \draw [ultra thin] ({0.242381},{-0.57735}) -- ({1.},{-0.57735}); 
 \draw [ultra thin] ({0.272857},{-0.866025})  -- ({1.182},{-0.866025}); 


\draw [ultra thin] ({0.7725},{0.2165063509}) -- ({1.03},{0.2886751346});
\draw [ultra thin] ({0.3175},{0.6495190528}) -- ({0.423333},{0.8660254038});
\draw [ultra thin] ({0.7725},{0.2165063509}) -- ({0.3175},{0.6495190528});
\draw [ultra thin] ({1.03},{0.2886751346})  -- ({0.423333},{0.8660254038}); 


\draw [ultra thin] ({1.},{0.317225}) -- ({1.25},{0.396532});
\draw [ultra thin] ({0.423333},{0.8660254038}) -- ({0.529167},{1.082531755});
\draw [ultra thin] ({1.},{0.317225}) -- ({0.423333},{0.8660254038});
\draw [ultra thin] ({1.25},{0.396532})  -- ({0.529167},{1.082531755}); 


\draw [ultra thin] ({1.},{0.634451}) -- ({1.146},{0.727081});
\draw [ultra thin] ({0.756667},{0.8660254038}) -- ({0.86714},{0.992465});
\draw [ultra thin] ({1.},{0.634451}) -- ({0.756667},{0.8660254038});
\draw [ultra thin] ({1.146},{0.727081})  -- ({0.86714},{0.992465}); 

\node [right = of b] {$b$};
\node [left = of a] {$a$};

\node [dot2=](abovea) at (0,{sqrt(3)/2}) {};
\node [above = of abovea] {$\abovesym{a}$};
\node [dot2=](aboveb) at (1,{sqrt(3)/2}) {};
\node [above = of aboveb] {$\abovesym{b}$};
\node [dot2=](undera) at (0,{-sqrt(3)/2}) {};
\node [below = of undera] {$\undersym{a}$};
\node [dot2=](underb) at (1,{-sqrt(3)/2}) {};
\node [below = of underb] {$\undersym{b}$};

\coordinate (aboveright) at ({0.40875000000000006}, {0.38});
\node [right = of aboveright] {$5$};
\coordinate (aboveleft) at ({0.24}, {0.35});
\node [left = of aboveleft] {$5$};
\coordinate (below) at  ({0.44}, {0.18});
\node [below = of below] {$5$};

\coordinate (notranjost) at ({0.36333333333333334}, {0.22}) {};
\node [above = of notranjost] {$3$};

\coordinate (B1) at ({0.56},{-0.22});
\node [above = of B1] {$7$};

\coordinate (B2) at ({0.7},{-0.51});
\node [above = of B2] {$10$};

\coordinate (B3) at ({0.694277},{-0.8});
\node [above = of B3] {$9$};

\coordinate (P2) at ({0.13},{-0.51});
\node [above = of P2] {$8$};

\coordinate (P3) at ({0.13},{-0.8});
\node [above = of P3] {$8$};

\coordinate (L11) at ({0.7},{0.5});
\node [left = of L11] {$7$};

\coordinate (L22) at ({0.8503241},{0.671687});
\node [left = of L22] {$8$};

\coordinate (L33) at ({0.99},{0.8});
\node [left = of L33] {$3$};

\coordinate (astar1) at ({0.218}, {0.173205-0.05}) {};
\coordinate (bstar1) at ({0.618+0.05}, {0.173205-0.05}) {};
\coordinate (cstar1) at ({0.254-0.05}, {0.519615+0.05}) {};

\node [left = of astar1] {$4$};
\node [right = of bstar1] {$4$};
\node [above = of cstar1] {$6$};
\coordinate (L1) at ({0.0},{0.35});
\node [above = of L1,fill=white] {$6$};
\draw [ultra thin] ({0.0675},{0.6495190528}) -- ({0.},{0.696535});

\draw [ultra thin] ({0.0225},{0.2165063509}) -- ({-0.0482574},{0.232178});
\draw [ultra thin] ({0.0675},{0.6495190528}) -- ({0.0225},{0.2165063509});
\draw [ultra thin] ({0.},{0.696535})  -- ({-0.0482574},{0.232178}); 
\fill[grey, opacity=0.3] ({0.0675},{0.6495190528}) -- ({0.},{0.696535}) -- ({-0.0482574},{0.232178}) -- ({0.0225},{0.2165063509}) -- cycle;

\end{footnotesize}
\end{tikzpicture}
}
\caption{Estimate $M < 102$ at $t = \frac{1}{11}$.}
\end{figure}
\end{frame}
%------------------------------------------------

%------------------------------------------------
\begin{frame}
\begin{block}{The Result}
We improved the upper bound by more than twice. We have shown that it
is not possible to draw a partial drawing of the complete graph on
102 or more points:
$$M < 102.$$

\end{block}
\begin{figure}[H]
\centering
\input{./tikz/number-line-5.tikz}
\end{figure}

We believe that the right estimate is much closer to 16 than to 102.
\end{frame}
%------------------------------------------------

\end{document} 
